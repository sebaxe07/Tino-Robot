\section{Background and Context}
This section will provide a comprehensive overview of the social robotics field and human-robot interaction research, establishing the foundation for understanding the Tino robot project's significance. The discussion will begin with a brief history of social robotics development and its evolution in academic and commercial applications. Following this, the section will introduce the Tino robot project as developed at Politecnico di Milano, detailing its role in advancing human-robot interaction research within the academic community. An overview of the original Tino robot will be presented, including its design philosophy, intended applications, and practical use cases in social interaction scenarios. Finally, the research context will be established within the AIRLab robotics laboratory, highlighting the interdisciplinary approach and research objectives that drive the Tino project development.

\section{Legacy Tino System Overview}
This section will present a detailed technical analysis of the original Tino robot architecture, providing the foundation for understanding the improvements made in Tino V2. The discussion will begin with a comprehensive description of the legacy system's hardware and software components, focusing on the Raspberry Pi-based control system and its inherent limitations in processing power and real-time performance. The original omnidirectional movement system utilizing the Triksta base will be examined, including its mechanical design, control algorithms, and operational characteristics. The basic sensor setup and capabilities of the legacy system will be detailed, covering the available sensing modalities and their integration into the robot's behavior system. Previous applications and use cases will be documented, demonstrating the robot's practical deployment in social interaction scenarios. Finally, the section will identify specific limitations and areas for improvement that motivated the development of Tino V2, including hardware constraints, software architecture issues, and performance bottlenecks.

\section{Project Motivation and Objectives}
This section will articulate the driving forces behind the Tino V2 development project and establish clear objectives for the research and development effort. The discussion will begin by identifying key limitations of the legacy system that necessitated a comprehensive redesign, including inadequate computational resources, limited sensor capabilities, and reliability issues. The need for improved localization and navigation capabilities will be established, highlighting the requirements for precise positioning in complex indoor environments and robust SLAM implementation. Requirements for VR integration and immersive interaction will be detailed, explaining how these capabilities extend the robot's potential applications and research value. Hardware reliability and performance enhancement needs will be discussed, focusing on the mechanical improvements required for sustained operation and user interaction. The section will conclude with an overview of research goals for advanced human-robot interaction, establishing the academic and practical contributions expected from the Tino V2 project.

\subsection{Technical Objectives}
This subsection will detail the specific technical goals that guide the Tino V2 development process. The migration from the Raspberry Pi platform to the NVIDIA Orin Nano will be explained, including the performance benefits and expanded capabilities this transition enables. The implementation of robust SLAM (Simultaneous Localization and Mapping) and sensor fusion techniques will be outlined, describing how these technologies address the localization challenges identified in the legacy system. The development of real-time human detection and pose estimation capabilities will be presented, explaining the integration of computer vision and machine learning technologies for enhanced human-robot interaction. The creation of VR-compatible control and interaction systems will be detailed, describing the software architecture and communication protocols required for seamless virtual reality integration. Finally, the improvement of mechanical reliability and performance will be discussed, covering hardware redesign efforts and component upgrades that enhance the robot's operational capabilities.

\subsection{Research Objectives}
This subsection will outline the research contributions and academic goals of the Tino V2 project. The investigation of hybrid localization approaches combining SLAM with Ultra-Wideband (UWB) positioning will be presented as a novel contribution to mobile robotics research. The development of atomic movement systems specifically designed for VR control will be detailed, explaining how this approach enables more natural and intuitive human-robot interaction paradigms. The exploration of advanced human-robot interaction paradigms will be discussed, focusing on how the enhanced sensing and control capabilities enable new forms of social interaction and user engagement. The study of real-time pose detection integration with robot behavior will be presented, demonstrating how human posture and gesture recognition can be incorporated into responsive robot behaviors for more natural interaction experiences.

\section{System Requirements and Constraints}
This section will establish the technical specifications and operational constraints that define the Tino V2 system design. Performance requirements for real-time operation will be detailed, including computational load specifications, response time requirements, and throughput expectations for various system components. Accuracy requirements for localization and human tracking will be specified, establishing the precision standards necessary for effective navigation and interaction capabilities. VR integration constraints and specifications will be outlined, including latency requirements, data exchange protocols, and synchronization standards required for seamless virtual reality operation. Physical constraints will be discussed, covering weight limitations, size restrictions, and power consumption requirements that influence the hardware design and component selection. Finally, reliability and robustness requirements for social interaction will be established, defining the operational standards necessary for safe and effective human-robot interaction in various environments.

\section{Thesis Structure and Contributions}
This section will provide readers with a clear roadmap of the thesis organization and highlight the key contributions of the research. An overview of the thesis organization and chapter flow will be presented, explaining how each chapter builds upon previous work and contributes to the overall narrative of the Tino V2 development project. The summary of main technical contributions will detail the engineering innovations and system improvements achieved in the project, including hardware upgrades, software architecture enhancements, and integration achievements. The summary of research contributions will highlight the academic value of the work, including novel approaches to mobile robot localization, human-robot interaction paradigms, and VR integration techniques. Key innovations and novel approaches introduced in the project will be emphasized, demonstrating how the work advances the state of the art in social robotics and mobile robot systems. Finally, the expected impact and applications of the developed system will be discussed, explaining how the Tino V2 platform enables new research opportunities and practical applications in human-robot interaction, education, and social robotics research.
