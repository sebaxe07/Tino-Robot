\section{Legacy System Analysis and Limitations}
This section will provide a comprehensive analysis of the original Tino robot system architecture, establishing the technical foundation that necessitated the migration to Tino V2. The legacy system evaluation will begin with a detailed examination of the Raspberry Pi-based control architecture, including its computational limitations for real-time processing, memory constraints that restricted simultaneous operation of multiple sensing modalities, and processing bottlenecks that prevented implementation of advanced computer vision algorithms. The original software architecture will be analyzed, covering the monolithic Python script design that combined all robot functionalities into single-threaded execution, leading to poor modularity, difficult debugging procedures, and limited scalability for adding new features. The communication system limitations will be discussed, including the lack of standardized message protocols, direct serial communication bottlenecks with Arduino subsystems, and absence of distributed processing capabilities. The sensor integration challenges will be examined, particularly the limited bandwidth for camera data processing, inadequate computational resources for SLAM implementation, and difficulties in achieving synchronized multi-sensor operation. Finally, the section will address reliability and maintenance issues, including system crashes due to resource exhaustion, difficulties in isolating and fixing component failures, and the challenges of remote debugging and system monitoring.

\section{ROS2 Architecture Design and Implementation}
This section will detail the comprehensive ROS2 architecture designed for Tino V2, explaining how the Robot Operating System 2 framework addresses the limitations identified in the legacy system. The ROS2 framework selection rationale will be presented first, covering its advantages in distributed computing, real-time performance capabilities, improved inter-process communication through DDS (Data Distribution Service), and enhanced security features compared to ROS1. The system architecture design principles will be explained, including the modular node-based approach that enables independent development and testing of subsystems, standardized message interfaces that facilitate component integration, and distributed processing capabilities that leverage the Orin Nano's enhanced computational resources. The communication infrastructure will be detailed, covering the publish-subscribe messaging paradigm, quality of service (QoS) policies for reliable data transmission, and the discovery mechanisms that enable automatic node detection and connection. The real-time considerations will be addressed, including deterministic message delivery, priority-based scheduling, and resource allocation strategies that ensure consistent performance for time-critical operations. The development and deployment advantages will be discussed, covering the improved debugging capabilities through ROS2 tools, enhanced logging and monitoring systems, and the simplified integration with external systems and simulation environments.

\section{Node Structure and Functionality}
This section will provide detailed descriptions of the individual ROS2 nodes that compose the Tino V2 system, explaining their specific responsibilities and interactions within the overall architecture. The gamepad\_node.py will be examined first, detailing its role in handling Xbox controller input with proper D-input to X-input conversion for Jetson compatibility, the implementation of the pulse-based command system that generates discrete movement commands for VR integration, and the enhanced command processing and error reporting mechanisms that improve control system robustness. The hardware\_interface\_node.py will be analyzed, covering its management of serial communication with all three Arduino systems (head, base, leg), the implementation of proper device symlinks for consistent Arduino connections, and the comprehensive debugging logs and status monitoring that enable effective system maintenance. The robot\_controller\_node.py functionality will be detailed, explaining its role as the central coordination node that manages all robot behaviors and movement commands, the implementation of the atomic movement system with 4-state leg and base control, and the synchronization mechanisms that ensure coordinated robot actions. The vr\_interface\_node.py will be examined, covering its handling of VR system integration and data exchange for Unity communication, the implementation of bidirectional audio communication systems, and the data recording and extraction capabilities that support VR system development and testing.

\section{Communication Protocols and Message Design}
This section will detail the communication infrastructure and message protocols developed for the Tino V2 system, ensuring reliable and efficient data exchange between all system components. The inter-node communication design will be explained first, covering the topic-based messaging system that enables decoupled component interaction, the service-call mechanisms for synchronous operations requiring immediate responses, and the action-based communication for long-running tasks with progress feedback. The custom message definitions will be detailed, including pose and localization messages that carry 3D position and orientation data with timestamp information, human detection messages that contain skeleton joint positions and confidence scores, and VR interface messages that facilitate bidirectional communication with Unity systems. The Arduino communication protocols will be examined, covering the enhanced serial communication protocols that provide reliable command transmission and status feedback, the implementation of command acknowledgment systems that ensure successful command execution, and the error handling mechanisms that detect and recover from communication failures. The data synchronization strategies will be discussed, including timestamp-based alignment of multi-sensor data, buffering mechanisms that handle varying data rates from different sensors, and the quality of service configurations that prioritize critical data streams while ensuring overall system stability.

\section{Integration with External Systems}
This section will explain how the ROS2 architecture facilitates integration with external systems and development tools, enhancing the overall capability and research value of the Tino V2 platform. The VR system integration will be detailed first, covering the ROS-TCP-Endpoint implementation that provides a communication bridge between ROS2 and Unity, the enhanced error handling mechanisms that ensure robust VR communication, and the data recording systems that capture comprehensive robot state information for VR development and analysis. The monitoring and debugging capabilities will be discussed, covering the comprehensive logging systems that track all node activities and system events, the diagnostic tools that provide real-time system health monitoring, and the remote access capabilities that enable development and troubleshooting via VNC connections. The extensibility features will be addressed, explaining how the modular architecture enables easy addition of new sensors and capabilities, the standardized interfaces that facilitate integration of third-party components, and the configuration management systems that allow dynamic parameter adjustment without system restart.

\section{Migration Benefits and System Improvements}
This section will quantify and analyze the improvements achieved through the migration from the legacy Raspberry Pi system to the ROS2-based Orin Nano architecture. The performance improvements will be detailed first, including the computational performance gains that enable real-time SLAM processing, computer vision algorithms, and multi-sensor fusion, the memory management improvements that allow simultaneous operation of multiple demanding processes, and the reduced latency in sensor data processing and command execution. The reliability enhancements will be examined, covering the improved system stability through modular architecture that isolates component failures, the enhanced error handling and recovery mechanisms that maintain system operation during partial failures, and the diagnostic capabilities that enable proactive maintenance and issue identification. The development efficiency improvements will be discussed, including the reduced development time for new features through standardized interfaces and modular design, the improved debugging capabilities that enable rapid issue identification and resolution, and the enhanced testing procedures that allow independent validation of system components. The operational advantages will be addressed, covering the simplified system startup and shutdown procedures, the improved remote monitoring and control capabilities, and the enhanced data logging and analysis tools that support research activities. Finally, the scalability benefits will be presented, explaining how the new architecture supports future expansion with additional sensors and capabilities, enables distributed processing across multiple computing platforms, and facilitates integration with cloud-based services and external research systems.
