\section{Kinematic Base Upgrade from Omnidirectional to Differential Drive}
This section will detail the comprehensive redesign of Tino's mobility system, transitioning from the problematic omnidirectional Triksta base to a robust differential drive architecture. The limitations of the original omnidirectional system will be analyzed first, covering the mechanical failures experienced with the omniwheel rollers that became squared due to Tino's 20kg weight, the dragging issues with the rear wheel that occurred during forward and turning movements, and the unreliable motor performance under the sustained loads required for social robot operation. The differential drive design rationale will be explained, including the simplified kinematics that eliminate the complexity of omnidirectional control while maintaining adequate maneuverability for social interaction scenarios, the improved weight distribution that reduces stress on individual components, and the enhanced reliability achieved through proven mechanical design principles. The mechanical implementation will be detailed, covering the construction of the T-structure using aluminum Item profiles that provide a dynamic and adjustable framework, the motor mounting system modifications required to accommodate the new differential drive configuration, and the wheel positioning optimization that achieves proper balance and traction for the robot's operational requirements. The control system adaptation will be examined, including the implementation of custom PID (Proportional-Integral-Derivative) controllers specifically designed for differential drive kinematics, the motor driver upgrade to the more powerful MDD10A units that can handle increased loads, and the command interface modifications that maintain compatibility with existing movement control systems while improving performance and reliability.

\section{Power Supply System Redesign for Orin Nano}
This section will present the comprehensive power system redesign required to support the NVIDIA Orin Nano platform and associated high-performance components. The power requirements analysis will be detailed first, covering the Orin Nano's 19V DC input requirement and power consumption characteristics that reach up to 2A during maximum computational load, the additional power needs for the Oak-D Pro camera and onboard router systems, and the total system power budget that necessitated complete redesign of the legacy Raspberry Pi power architecture. The DC-DC converter implementation will be explained, including the selection and testing of the Oumefar 12V to 19V step-up converter that provides stable power delivery, the power efficiency analysis that demonstrates optimal battery utilization, and the thermal management considerations that ensure reliable operation under sustained loads. The battery system optimization will be examined, covering the consolidation from four separate battery systems to three integrated power sources, the 5200mAh 80C 11.1V 57.72Wh battery specification that provides approximately 1.37 hours of operation at maximum load, and the realistic operational time estimates of 2-3 hours under typical social interaction scenarios. The cable harness redesign will be detailed, including the removal of legacy USB-A and USB-C connections that were used for Raspberry Pi power delivery, the implementation of proper 12V input distribution and 19V DC jack connectivity, and the integration of the 12V to 5V converter that powers the onboard router and camera systems independently, providing flexibility for future system expansions and reducing the computational load on the main platform.

\section{Stewart Platform Head Mechanism Improvements}
This section will document the iterative design improvements made to Tino's Stewart platform head mechanism to address reliability issues and enhance performance under operational loads. The original system limitations will be analyzed first, covering the servo axis misalignment problems that created excessive stress on servo motors during head movements, the structural flex issues in the connecting arms that caused mechanical instability and reduced precision, and the repeated arm failures that occurred due to inadequate load distribution and material selection. The first design iteration will be detailed, including the servo axis alignment improvement that redirected forces through the head structure rather than the servo mechanisms, the 3D printed PLA arm replacement with enhanced geometry for improved load distribution, and the initial performance evaluation that showed reduced servo stress but continued structural flex issues. The final design implementation will be examined, covering the adoption of rod end (heim joints) on both ends of each Stewart platform arm to eliminate binding and allow free rotation, the combination of 3D printed components with metal heim joints that provides optimal balance between cost and performance, and the mechanical trade-offs including acceptable head wobble during stationary periods that may actually enhance the robot's expressive capabilities. The performance validation will be discussed, including load testing that demonstrates improved reliability under operational conditions, the movement precision evaluation that shows maintained accuracy despite the mechanical improvements, and the longevity testing that validates the enhanced design's suitability for extended social interaction scenarios.

\section{Camera Integration and Mounting Solutions}
This section will detail the comprehensive camera integration system developed to address the unique challenges of mounting sophisticated sensing equipment within Tino's soft fabric structure. The mounting system design will be explained first, covering the tripod-based camera support system that provides stable mounting for the Oak-D Pro camera, the bracket design that ensures proper camera alignment and minimizes vibration during robot movement, and the integration with the existing Stewart platform head that allows synchronized camera and head movements. The fabric integration challenges will be analyzed, including the camera visibility requirements that necessitate fabric modification without compromising Tino's aesthetic appeal, the heat dissipation needs of the Oak-D Pro camera that require ventilation considerations, and the protection requirements that shield sensitive camera components from physical damage during social interactions. The camera shell development will be detailed, covering the custom enclosure design that provides protection while maintaining cooling airflow, the velcro attachment system that secures fabric positioning without interfering with camera operation, and the mesh covering implementation that conceals the camera from casual observation while maintaining full operational capability. The field of view optimization will be examined, including the fabric positioning strategies that prevent interference with camera sensing, the testing procedures that validate optimal camera performance under various fabric configurations, and the reliability evaluation that ensures consistent operation throughout extended social interaction sessions.

\section{Audio System Integration}
This section will present the comprehensive audio system implementation that enables bidirectional communication capabilities for VR integration and enhanced human-robot interaction. The hardware component selection will be detailed first, covering the iTalk-01 omnidirectional microphone specification and mounting considerations within the fabric head structure, the speaker system selection and placement optimization that provides clear audio output without interfering with other robot systems, and the audio processing requirements that enable real-time communication with VR systems. The integration challenges will be analyzed, including the acoustic isolation needed to prevent feedback between microphone and speakers, the cable routing through the robot's structure that maintains mechanical flexibility while ensuring reliable connections, and the power management considerations that integrate audio components with the overall system power budget. The software implementation will be examined, covering the audio\_node.py and audio\_loopback.py ROS2 nodes that handle audio capture and playback, the bidirectional communication protocols that enable seamless VR audio integration, and the audio processing algorithms that ensure high-quality sound transmission and reception. The performance validation will be discussed, including audio quality testing that demonstrates suitable performance for human-robot communication, latency measurements that verify real-time communication capabilities, and integration testing that validates seamless operation with the VR system and overall robot behavior control.

\section{Mechanical Reliability Improvements and Testing}
This section will document the systematic approach to identifying and resolving mechanical reliability issues that affected the legacy Tino system and the validation procedures used to ensure improved performance in Tino V2. The failure analysis methodology will be explained first, covering the systematic documentation of component failures during development and testing, the root cause analysis procedures that identified design weaknesses and operational stress factors, and the prioritization of improvements based on criticality and impact on robot operation. The wheel system improvements will be detailed, including the plastic wheel hub failure analysis that led to hot glue reinforcement as an interim solution, the tire de-beading issues caused by robot weight and the hot glue filling solution that restored proper tire-to-rim interface, and the wheel bumper implementation that prevents fabric entanglement during robot movement. The structural enhancements will be examined, covering the aluminum profile framework that provides improved rigidity and adjustability compared to the original design, the motor bracket modifications that ensure proper alignment and reduce mechanical stress, and the fastener and connection improvements that enhance overall system reliability. The validation testing procedures will be discussed, including the systematic load testing that verifies component performance under operational conditions, the endurance testing that demonstrates sustained operation capabilities, and the performance monitoring that tracks system health during extended operational periods. Finally, the preventive maintenance protocols will be presented, covering the inspection procedures that enable early detection of potential issues, the component replacement schedules that prevent unexpected failures, and the documentation systems that track system performance and maintenance history for continuous improvement of mechanical reliability.
