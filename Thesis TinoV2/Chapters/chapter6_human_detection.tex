\section{YOLOv11 Pose Detection Implementation with TensorRT Optimization}
This section details the implementation of YOLOv11 pose detection system optimized for real-time performance on the NVIDIA Orin Nano platform using pre-trained models. The YOLOv11 architecture selection will be explained first, covering the advantages of using the latest YOLO iteration for pose estimation tasks, including improved accuracy in detecting multiple humans simultaneously and optimized network architecture that balances detection accuracy with computational efficiency for embedded platforms. The pre-trained model utilization will be detailed, including the selection of the yolo11n-pose.pt model that provides an optimal balance between accuracy and computational requirements, the model format conversion from PyTorch (.pt) to ONNX (.onnx) format for cross-platform compatibility, and the final optimization to TensorRT engine (.engine) format that maximizes inference performance on the Orin Nano's GPU. The TensorRT optimization process will be examined, covering the engine generation procedures that optimize the neural network for the specific hardware platform and the memory allocation strategies that ensure efficient GPU utilization during real-time operation. The implementation architecture will be discussed, including the ROS2 node design that subscribes to camera topics from the Oak-D Pro and publishes human detection results, and the message publishing system that provides skeleton joint information with confidence scores for other system components.

\section{Stereo Depth Integration for 3D Human Positioning}
This section will present the integration of stereo depth information with 2D pose detection to achieve 3D human positioning capabilities using the Oak-D Pro camera system. The depth data utilization will be detailed first, covering how the stereo camera system provides depth information at detected keypoint locations, the depth value extraction process that determines 3D coordinates for each detected joint, and the coordinate system transformation from camera frame to robot coordinate frame. The 3D positioning methodology will be explained, including the process of combining 2D joint detections with corresponding depth values to create 3D skeleton representations and the real-time processing requirements that maintain system responsiveness while providing human positioning information. The practical implementation will be discussed, including how depth measurement works at varying distances and the integration with the robot's localization system to provide human positions relative to the robot's coordinate frame.

\section{Real-time Skeleton Tracking with 17 Key Body Joints}
This section will detail the skeleton tracking implementation that extracts and processes 17 standard COCO keypoints for human posture detection. The keypoint detection framework will be explained first, covering the 17 keypoints detected by YOLOv11 including nose, eyes, ears, shoulders, elbows, wrists, hips, knees, and ankles, and the confidence scoring system that indicates detection reliability for each joint. The data processing pipeline will be detailed, including the extraction of keypoint coordinates and confidence scores from YOLOv11 output, the organization of joint information into structured skeleton representations, and the real-time publishing of skeleton data through ROS2 topics. The message format and data flow will be discussed, including the custom ROS2 message structures that transmit skeleton data with timestamps and the integration with other system components that can utilize human pose information for robot operation and VR data recording.

\section{Performance Optimization and Real-time Operation}
This section will present the optimization strategies implemented to achieve real-time performance of the human detection system on the Orin Nano platform. The computational optimization techniques will be detailed first, covering the TensorRT engine utilization that maximizes GPU inference performance and the memory management strategies that prevent resource exhaustion during continuous operation. The real-time performance characteristics will be explained, including the frame rate capabilities achieved by the optimized system and the processing latency from camera input to pose detection output. The system integration optimization will be examined, covering how the pose detection system operates alongside other robot functions including SLAM, localization, and movement control without creating performance bottlenecks. The practical performance evaluation will be discussed, including testing under various operational scenarios and system stability during extended operation periods.

\section{Integration with System Architecture}
This section will detail the integration of human detection with Tino's overall system architecture and coordination with other robot components. The data flow architecture will be explained first, covering how human pose detection results are published through ROS2 topics and made available to other system components including robot controller nodes and navigation systems. The system coordination will be discussed, including how pose detection operates in parallel with other robot functions such as localization, movement control, and audio processing without creating performance bottlenecks. The ROS2 integration implementation will be examined, covering the message publishing system that provides skeleton joint information with timestamps and the node architecture that ensures reliable data transmission to other system components. Finally, the practical benefits will be addressed, covering how real-time human detection enhances the robot's operational awareness and enables improved human-robot interaction capabilities through better understanding of human presence and positioning.
