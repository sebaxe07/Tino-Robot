\section{Localization System Performance Evaluation}
This section will present the evaluation results of Tino's localization systems, comparing pure SLAM performance with the implemented UWB sensor fusion approach. The RTAB-Map SLAM baseline evaluation will be detailed first, covering the systematic testing conducted at four fixed positions with multiple iterations, the significant drift issues discovered with maximum deviations of 1.20 meters, and the need for manual relocalization through spinning movements that proved impractical for VR users without technical expertise. The UWB sensor fusion implementation assessment will be examined, including the positioning system integration that provides improved accuracy with position coordinates showing better consistency across multiple trials, the RTAB-Map orientation preservation that maintained reliable heading information, and the significant improvement in positioning stability with the combined UWB position and SLAM orientation approach. The experimental methodology will be discussed, covering the controlled testing environment with marked floor positions for measurement points, the multiple iteration protocol that provided comparison data, and the systematic data collection that revealed both system capabilities and limitations. The performance comparison analysis will be addressed, including position accuracy measurements showing UWB coordinates with improved consistency compared to pure SLAM, orientation reliability maintained through RTAB-Map integration, and the practical implications for VR integration requiring more reliable positioning without manual intervention.

\section{System Limitations and Identified Challenges}
This section will provide a comprehensive analysis of the identified system limitations and technical challenges encountered during development and testing. The localization system constraints will be detailed first, covering the RTAB-Map dependency on visual features that requires careful environment preparation with sufficient landmarks, the drift accumulation issues that necessitated the UWB sensor fusion implementation, and the map alignment challenges that require manual angle offset adjustments for proper Unity integration. The hardware reliability concerns will be examined, including the mechanical stress points identified during testing such as wheel hub failures requiring temporary repairs, the power system limitations that affect extended operation capabilities, and the component integration challenges that impact overall system robustness. The computational resource limitations will be discussed, covering the NVIDIA Orin Nano platform constraints that require careful optimization of concurrent processes, the memory management requirements for simultaneous SLAM, pose detection, and VR communication operations, and the thermal considerations during extended operation periods. The network communication challenges will be addressed, including the UDP packet loss scenarios that require robust error handling, the bandwidth limitations when transmitting high-frequency pose and skeleton data, and the latency variations that affect VR user experience quality. The VR system integration constraints will be covered, including the Unity development complexity for implementing robust robot communication, the synchronization challenges between virtual and physical environments, and the user experience limitations when technical expertise is required for system recovery or troubleshooting.

\section{Future Improvements and Research Directions}
This section will outline the comprehensive roadmap for future system enhancements and research opportunities identified through the evaluation process. The localization system enhancements will be detailed first, covering the potential integration of additional UWB anchors for improved positioning accuracy and coverage area expansion, the exploration of advanced sensor fusion techniques combining IMU data with existing UWB and visual systems, and the development of automatic map alignment algorithms that eliminate manual angle offset requirements. The mechanical system improvements will be examined, including the replacement of failure-prone components with more robust alternatives, the implementation of predictive maintenance capabilities through sensor monitoring, and the exploration of advanced actuator systems that provide smoother movement profiles and reduced mechanical stress. The computational optimization opportunities will be discussed, covering the potential migration to more powerful embedded platforms as they become available, the implementation of distributed processing architectures that leverage multiple computational units, and the development of adaptive resource management systems that dynamically allocate processing power based on operational requirements. The VR integration advancement possibilities will be addressed, including the development of more sophisticated interaction paradigms that leverage advanced pose detection capabilities, the implementation of haptic feedback systems that enhance user immersion and control precision, and the exploration of multi-user VR environments that support collaborative robot interaction scenarios. The research application extensions will be covered, including the development of comprehensive datasets for human-robot interaction research, the implementation of machine learning systems that adapt robot behavior based on user interaction patterns, and the exploration of social robotics applications that leverage the established VR integration framework. Finally, the long-term vision will be presented, covering the potential evolution toward fully autonomous social interaction capabilities, the integration with broader smart environment systems, and the development of replicable platforms that enable wider adoption of VR-integrated social robotics research.

\section{Conclusions and Lessons Learned}
This section will synthesize the key findings from the Tino V2 development process, highlighting significant achievements and valuable insights for future robotics research. The technical achievement summary will be presented first, covering the successful migration from legacy Raspberry Pi architecture to modern ROS2-based systems on NVIDIA Orin Nano, the implementation of sensor fusion combining UWB positioning with RTAB-Map orientation that addresses localization drift issues, and the development of atomic movement systems that enable natural VR interaction through discrete, completion-guaranteed operations. The methodological insights will be detailed, including the importance of systematic testing protocols for identifying system limitations like SLAM drift issues, the value of modular architecture design that enables independent development of localization, movement, detection, and communication subsystems, and the critical role of practical testing in revealing real-world challenges such as mechanical reliability concerns. The interdisciplinary integration lessons will be examined, covering the challenges and benefits of combining computer vision, robotics, networking, and VR technologies in a cohesive system, the importance of maintaining real-time performance requirements across all subsystems, and the value of comprehensive documentation and monitoring systems for troubleshooting complex multi-component interactions. The research contribution significance will be discussed, including the advancement of VR-integrated social robotics through practical implementation of bidirectional communication systems, the development of atomic movement architectures that ensure predictable robot behavior for remote users, and the demonstration of sensor fusion approaches that combine complementary positioning technologies. The broader implications will be addressed, covering the potential impact on social robotics research through demonstrated VR integration capabilities, the contribution to human-robot interaction studies through comprehensive data recording and analysis tools, and the advancement of embedded robotics development through optimization of complex algorithms on resource-constrained platforms. Finally, the future outlook will be presented, covering the foundation established for advanced social interaction research, the potential for broader adoption of VR-integrated robotics platforms, and the continued evolution toward more sophisticated and accessible human-robot interaction systems that bridge physical and virtual environments effectively.
