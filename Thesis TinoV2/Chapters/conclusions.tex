\section{Summary of Achievements}
This section will synthesize the major achievements accomplished through the Tino V2 development project, highlighting the successful transformation from a legacy system to a modern, capable social robotics platform. The technological transformation will be summarized first, covering the complete migration from Raspberry Pi-based architecture to NVIDIA Orin Nano systems that provided substantial improvements in computational capability and real-time performance, the successful integration of advanced sensing technologies including Oak-D Pro depth cameras and UWB positioning systems that enhanced the robot's perception and localization capabilities, and the implementation of modern ROS2-based software architecture that replaced monolithic legacy code with modular, maintainable systems. The sensor fusion success will be highlighted, covering the development and validation of UWB-SLAM fusion that addressed critical indoor localization drift issues while maintaining real-time performance requirements, the achievement of centimeter-level positioning accuracy that enables reliable VR integration and precise navigation, and the demonstration of robust localization performance across various environmental conditions and operational scenarios. The VR integration accomplishments will be detailed, covering the successful implementation of bidirectional communication systems that enable natural remote interaction through Unity-based VR environments, the development of atomic movement architectures that provide predictable and intuitive robot control for non-technical users, and the creation of comprehensive data recording and analysis systems that support ongoing research in human-robot interaction. The human detection achievements will be presented, covering the successful implementation of real-time pose estimation using optimized YOLOv11 models that provide 17-keypoint skeletal tracking with depth information, the achievement of simultaneous SLAM and human detection processing on embedded hardware, and the integration of detection results with robot control systems for responsive human-robot interaction.

\section{Technical Contributions and Innovation}
This section will detail the specific technical contributions made to the field of social robotics and the innovative approaches developed during this thesis work. The sensor fusion methodology contribution will be established first, covering the practical demonstration of UWB-visual SLAM integration that addresses common indoor robotics localization challenges, the development of coordinate system alignment procedures that enable reliable fusion of absolute and relative positioning data, and the validation of sensor fusion approaches that maintain real-time performance while improving localization accuracy and reliability. The atomic movement architecture contribution will be detailed, covering the development of discrete state-based control systems that enable predictable VR interaction with physical robots, the implementation of synchronization mechanisms that coordinate multiple robot subsystems for natural movement patterns, and the creation of command queuing and conflict resolution systems that handle complex user interactions safely and effectively. The real-time optimization contributions will be examined, covering the successful implementation of simultaneous SLAM, human detection, and VR communication on embedded hardware through careful architectural design and optimization strategies, the development of parallel processing approaches that distribute computational load effectively across multiple ROS2 nodes, and the achievement of low-latency communication between VR and robotics systems that enables responsive remote interaction. The integration methodology contributions will be addressed, covering the development of comprehensive testing and validation procedures that ensure reliable multi-component system operation, the creation of modular software architectures that enable independent development and debugging of complex robotics systems, and the implementation of robust error handling and recovery mechanisms that maintain system stability under challenging operational conditions.

\section{Research Impact and Broader Implications}
This section will examine the broader research impact and implications of the Tino V2 project for the field of social robotics and related disciplines. The social robotics advancement will be assessed first, covering how the demonstrated VR integration capabilities advance the state of the art in telepresence robotics and remote human-robot interaction, the contribution to understanding of user experience design principles for VR-controlled robotics systems, and the validation of practical approaches for implementing complex social robotics capabilities on accessible hardware platforms. The sensor fusion research impact will be detailed, covering the practical demonstration of indoor localization solutions that address common challenges faced by mobile robots in GPS-denied environments, the contribution to understanding of sensor fusion trade-offs and optimization strategies for real-time embedded applications, and the validation of UWB positioning as a complementary technology for visual SLAM systems. The embedded AI research contributions will be examined, covering the successful demonstration of real-time human pose detection on embedded hardware through systematic optimization approaches, the contribution to understanding of model optimization trade-offs for practical robotics applications, and the validation of TensorRT and similar optimization frameworks for enabling advanced AI capabilities on resource-constrained platforms. The interdisciplinary research impact will be addressed, covering the contribution to understanding of effective integration strategies for combining computer vision, robotics, networking, and VR technologies in cohesive systems, the advancement of user-centered design principles for complex technical systems that must be accessible to non-technical users, and the demonstration of practical approaches for bridging virtual and physical environments in meaningful and reliable ways.

\section{Limitations and Challenges}
This section will provide honest assessment of the limitations encountered and challenges that remain unresolved in the current Tino V2 implementation. The technical limitations will be acknowledged first, covering the remaining dependence on UWB anchor infrastructure that limits deployment flexibility and requires careful setup and calibration procedures, the computational constraints that, while significantly improved, still impose trade-offs between system capability and real-time performance requirements, and the environmental dependencies that affect system performance under certain lighting conditions, visual feature scarcity, or UWB signal obstruction scenarios. The mechanical system limitations will be addressed, covering the ongoing reliability challenges with the Stewart platform head mechanism despite improvements with heim joint implementations, the battery life constraints that limit extended operation periods and require careful power management, and the mechanical wear issues that affect long-term system reliability and require ongoing maintenance and replacement of components. The VR integration limitations will be examined, covering the network latency and reliability dependencies that can affect user experience during remote operation, the learning curve required for users to effectively operate the VR control system despite efforts to create intuitive interfaces, and the limited environmental feedback that restricts user awareness of the robot's physical surroundings and potential obstacles. The scalability challenges will be discussed, covering the manual calibration and setup requirements that limit rapid deployment in new environments, the complexity of system integration that requires technical expertise for maintenance and troubleshooting, and the current limitation to laboratory and controlled environments that restricts broader real-world deployment and validation.

\section{Future Work and Research Directions}
This section will outline promising directions for future research and development that build upon the foundations established in the Tino V2 project. The localization enhancement opportunities will be presented first, covering the potential for implementing fully autonomous UWB anchor deployment and calibration systems that reduce setup complexity and increase deployment flexibility, the exploration of additional sensor modalities such as LiDAR or thermal imaging that could further improve localization robustness and environmental perception, and the development of adaptive sensor fusion algorithms that dynamically adjust fusion parameters based on environmental conditions and sensor reliability. The VR interaction advancement possibilities will be detailed, covering the potential for implementing more sophisticated haptic feedback systems that provide users with tactile information about the robot's environment and interactions, the development of multi-user VR systems that enable collaborative control and interaction with the robot platform, and the exploration of augmented reality integration that combines real-world robot camera feeds with virtual overlay information for enhanced user experience. The human-robot interaction research directions will be examined, covering the potential for implementing more advanced social behavior algorithms that respond appropriately to detected human emotions and social cues, the development of learning systems that adapt robot behavior based on individual user preferences and interaction patterns, and the exploration of multi-robot coordination capabilities that enable collaborative social robotics applications. The broader deployment opportunities will be addressed, covering the potential for adapting the developed technologies for healthcare applications such as elderly care and patient monitoring, the exploration of educational applications that leverage the VR integration for remote learning and interaction, and the investigation of commercial applications that could benefit from reliable telepresence robotics with advanced perception and interaction capabilities.

\section{Final Reflections and Conclusions}
This section will provide final reflections on the Tino V2 project and draw overall conclusions about the research achievements and their significance for the field of social robotics. The project success assessment will be established first, covering the achievement of all major technical objectives including successful system migration, sensor fusion implementation, VR integration, and human detection capabilities, the validation of key hypotheses about the feasibility and effectiveness of combining multiple advanced technologies in a cohesive social robotics platform, and the demonstration of practical approaches for addressing common challenges in indoor mobile robotics and human-robot interaction. The methodological insights will be reflected upon, covering the value of systematic technology evaluation and selection processes for complex multi-component systems, the importance of modular architecture design for enabling effective development and debugging of advanced robotics systems, and the critical role of user-centered design principles for creating accessible and effective human-robot interaction systems. The research contribution significance will be assessed, covering how the Tino V2 project advances the state of the art in VR-integrated social robotics through practical demonstration of reliable technologies and effective integration strategies, the contribution to broader understanding of sensor fusion approaches for indoor robotics applications, and the validation of embedded AI optimization techniques for real-time perception and interaction capabilities. The legacy and impact considerations will be addressed, covering the foundation established for future social robotics research through open-source code, comprehensive documentation, and validated system architectures, the potential for broader adoption of demonstrated technologies and approaches by other researchers and developers, and the continued evolution toward more sophisticated and accessible human-robot interaction systems that effectively bridge virtual and physical environments for meaningful social applications.
