\section{Localization System Performance Analysis}
This section will present a comprehensive analysis of the localization system performance, comparing the sensor fusion approach against baseline SLAM-only operation to demonstrate the improvements achieved through UWB integration. The experimental methodology will be established first, covering the controlled testing environment setup using fixed reference points marked on the laboratory floor, the systematic data collection protocols that captured position estimates from both SLAM-only and UWB-fused systems during identical movement patterns, and the statistical analysis methods used to quantify localization accuracy and consistency. The SLAM-only baseline performance will be analyzed, covering the drift characteristics observed during extended operation where position estimates deviated by up to 1.2 meters from true positions at fixed reference points, the orientation accuracy that remained reliable even when position estimates suffered significant drift, and the specific scenarios that triggered substantial localization errors including proximity to walls and areas with limited visual features. The sensor fusion performance improvements will be detailed, covering the substantial reduction in position error achieved through UWB integration where maximum deviations were reduced to centimeter-level accuracy, the improved consistency of position estimates across multiple test runs at identical reference points, and the enhanced reliability of localization during challenging scenarios that previously caused significant SLAM drift. The coordinate system alignment results will be presented, covering the successful integration of UWB global coordinates with RTABMap local coordinate systems, the accuracy of coordinate transformations validated through systematic position measurements at known reference points, and the stability of alignment parameters that maintained consistent coordinate mapping throughout extended testing sessions.

\section{Human Detection and Pose Estimation Performance}
This section will analyze the performance of the YOLOv11-based human detection and pose estimation system, focusing on accuracy, processing speed, and reliability under various operational conditions. The detection accuracy analysis will be presented first, covering the successful implementation of 17-keypoint human pose detection that provides comprehensive skeletal tracking information, the depth integration performance that enables accurate 3D position estimation for detected humans using Oak-D Pro stereo camera data, and the tracking consistency that maintains stable person identification across consecutive frames despite variations in pose and position. The real-time performance evaluation will be detailed, covering the achievement of real-time inference speeds through TensorRT optimization that enables simultaneous SLAM and human detection processing, the computational resource utilization analysis that demonstrates efficient operation within the constraints of the Orin Nano platform, and the system responsiveness that provides immediate detection results for integration with robot control systems. The robustness testing results will be examined, covering the system performance under varying lighting conditions and camera perspectives that demonstrated reliable detection across realistic operational scenarios, the handling of multiple people in the camera field of view with appropriate tracking and identification capabilities, and the error handling performance that maintains system stability when detection temporarily fails or produces uncertain results. The integration effectiveness will be addressed, covering the successful coordination between human detection and robot behavior systems that enables responsive human-robot interaction, the coordinate system accuracy that properly aligns detected human poses with robot and world coordinate frames, and the data quality that provides meaningful input for research analysis and VR system integration.

\section{VR Integration and User Experience Analysis}
This section will evaluate the effectiveness of the VR integration system and the user experience provided by the atomic movement control architecture. The VR communication performance will be analyzed first, covering the low-latency data exchange achieved between Unity and ROS2 systems that enables responsive robot control from VR environments, the bidirectional audio communication quality that supports natural conversation between VR users and people near the robot, and the system reliability that maintains stable communication despite network variations and computational load changes. The atomic movement system evaluation will be detailed, covering the successful implementation of the four-state control architecture that provides predictable and intuitive robot behavior for VR users, the synchronization accuracy between leg and base movements that creates natural-looking robot locomotion, and the command queuing effectiveness that handles user inputs during ongoing movements without losing commands or creating conflicts. The user interface effectiveness will be examined, covering the intuitive VR control interfaces that enable effective robot operation without requiring technical expertise, the feedback systems that provide users with clear information about robot state and movement progress, and the safety mechanisms that prevent dangerous or conflicting robot behaviors during remote operation. The overall user experience will be assessed, covering the natural interaction feel achieved through careful timing and coordination of robot movements, the immersion quality that makes users feel genuinely present in the robot's environment, and the accessibility features that accommodate users with varying levels of VR experience and technical background.

\section{System Reliability and Performance Metrics}
This section will present comprehensive analysis of overall system reliability, computational performance, and operational characteristics under extended use conditions. The system stability analysis will be established first, covering the uptime performance achieved during extended testing sessions with the ROS2 architecture demonstrating robust operation without memory leaks or performance degradation, the error recovery capabilities that maintain system functionality when individual components experience temporary failures, and the graceful degradation behavior that preserves core functionality when non-critical subsystems encounter problems. The computational performance evaluation will be detailed, covering the resource utilization analysis that demonstrates efficient operation within the constraints of the Orin Nano platform while simultaneously running SLAM, human detection, and VR communication systems, the real-time performance maintenance that preserves responsive operation despite varying computational loads, and the power consumption characteristics that enable practical battery-powered operation for meaningful durations. The integration robustness will be examined, covering the communication reliability between ROS2 nodes that maintains consistent data flow despite network variations and system load changes, the sensor fusion stability that provides consistent localization performance across different environmental conditions and operational scenarios, and the modular architecture effectiveness that enables independent debugging and optimization of individual subsystems. The operational flexibility will be addressed, covering the ease of switching between mapping and localization modes that enables efficient system deployment in new environments, the configuration management effectiveness that allows system optimization for different operational requirements, and the maintenance and troubleshooting capabilities that support ongoing research and development activities with minimal downtime and technical expertise requirements.

\section{Comparative Analysis and Lessons Learned}
This section will synthesize the evaluation results to provide comparative analysis against baseline systems and extract key lessons learned from the implementation and testing process. The performance improvement quantification will be presented first, covering the measurable improvements achieved through the migration from Raspberry Pi to Orin Nano architecture including processing speed increases and capability expansion, the localization accuracy improvements demonstrated through UWB sensor fusion compared to SLAM-only operation, and the system reliability enhancements achieved through modular ROS2 architecture compared to legacy monolithic implementations. The technology validation results will be detailed, covering the successful validation of RTABMap SLAM for indoor social robot applications despite challenges with visual feature scarcity and dynamic environments, the effectiveness of UWB positioning for absolute coordinate reference despite occasional signal degradation and non-line-of-sight conditions, and the practical feasibility of real-time human pose detection on embedded platforms through careful optimization and model selection. The implementation insights will be examined, covering the critical importance of systematic testing and calibration procedures for achieving reliable sensor fusion performance, the value of modular software architecture for enabling efficient debugging and system optimization, and the significance of user-centered design principles for creating effective VR interaction systems. The research contribution significance will be assessed, covering the advancement of practical sensor fusion approaches that address common indoor robotics challenges, the development of atomic movement architectures that enable predictable VR-robot interaction, and the demonstration of comprehensive integration strategies that combine multiple advanced technologies in a cohesive, reliable robotic system suitable for social robotics research and applications.
