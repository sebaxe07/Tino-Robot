\section{Project Overview}
This section will introduce the Tino V2 social robotics project, providing context for the comprehensive system redesign and modernization effort undertaken during this thesis work. The project motivation will be presented first, covering the evolution from the original Tino social robot to the need for substantial technological upgrades that address limitations in processing power, localization accuracy, and remote interaction capabilities. The thesis scope will be defined, encompassing the complete migration from legacy Raspberry Pi architecture to modern NVIDIA Orin Nano systems, the implementation of advanced SLAM and sensor fusion technologies, the development of VR integration systems for remote human-robot interaction, and the redesign of mechanical systems to support enhanced functionality. The problem statement will be outlined, identifying key challenges including the unreliable localization performance of the original system that suffered from significant drift issues, the computational limitations that prevented real-time implementation of advanced computer vision algorithms, the lack of remote operation capabilities that limited the robot's utility for telepresence applications, and the mechanical reliability issues that affected consistent operation during extended use. The research objectives will be clearly defined, covering the development of robust localization systems that combine visual SLAM with UWB positioning for drift-free navigation, the implementation of real-time human detection and pose estimation using optimized YOLOv11 models, the creation of seamless VR integration that enables natural remote control through Unity-based virtual environments, and the enhancement of mechanical systems to support reliable long-term operation with improved differential drive locomotion.

\section{Social Robotics Context}
This section will establish the broader context of social robotics research and position the Tino V2 project within current technological trends and academic research directions. The social robotics landscape will be examined, covering the growing importance of robots designed for human interaction in domestic, healthcare, and educational environments, the role of telepresence robotics in enabling remote social connections, and the integration of virtual and augmented reality technologies to enhance human-robot interaction modalities. The technological convergence will be discussed, highlighting how advances in embedded AI processing, real-time computer vision, and wireless communication technologies have created new possibilities for sophisticated social robots, the emergence of ROS2 as a standardized framework for modern robotics development, and the increasing accessibility of high-performance sensors like depth cameras and UWB positioning systems. The research contribution significance will be established, covering how the Tino V2 project advances the state of the art in VR-integrated social robotics through practical implementation of bidirectional communication systems, the development of atomic movement architectures that ensure predictable robot behavior for remote users, and the demonstration of sensor fusion approaches that address common localization challenges in indoor environments.

\section{Technical Challenges and Innovation}
This section will detail the specific technical challenges addressed in this thesis and the innovative solutions developed to overcome them. The localization challenge will be examined first, covering the fundamental problem of indoor robot navigation where GPS is unavailable and traditional visual SLAM systems suffer from drift accumulation, the need for absolute positioning accuracy to enable reliable VR integration where virtual and physical environments must remain synchronized, and the implementation of sensor fusion combining RTABMap visual SLAM with UWB positioning to achieve robust localization performance. The real-time processing challenge will be addressed, covering the computational demands of simultaneous SLAM processing, human pose detection, and VR communication on embedded hardware, the optimization strategies employed including TensorRT acceleration for YOLOv11 models and efficient ROS2 node architecture design, and the balance achieved between system performance and power consumption requirements. The VR integration challenge will be explored, covering the development of atomic movement systems that ensure consistent mapping between virtual user actions and physical robot responses, the implementation of bidirectional audio communication for natural telepresence interaction, and the creation of Unity-based VR environments that provide intuitive control interfaces for non-technical users.

\section{Thesis Structure and Contributions}
This section will outline the organization of this thesis and summarize the key contributions made to the field of social robotics. The thesis organization will be presented, covering how Chapter 2 establishes the background and related work in SLAM, sensor fusion, human detection, and VR integration technologies, Chapter 3 details the conceptual work including technology selection criteria and system architecture design decisions, Chapter 4 presents the implementation of all major system components including hardware integration and software development, Chapter 5 provides evaluation of system performance and comparison with baseline approaches, and Chapter 6 concludes with discussion of achievements, limitations, and future research directions. The primary contributions will be summarized, covering the successful integration of UWB positioning with visual SLAM to address indoor localization drift issues, the development of atomic movement control architecture that enables predictable VR interaction with physical robots, the implementation of real-time human pose detection on embedded hardware using optimized YOLOv11 models, the creation of comprehensive VR integration systems that support bidirectional communication and natural user interfaces, and the demonstration of practical sensor fusion approaches that combine complementary positioning technologies. The broader impact will be discussed, covering how these contributions advance the field of social robotics by providing validated approaches for reliable indoor navigation, remote interaction, and human detection that can be adopted by other researchers and extended for various social robotics applications.
