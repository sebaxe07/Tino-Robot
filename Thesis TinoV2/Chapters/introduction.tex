\section{Social Robotics and Human-Robot Interaction}
The field of social robotics has emerged from the fundamental human need for meaningful interaction and connection. As society increasingly integrates technology into daily life, the motivation for developing robots capable of effective social interaction becomes paramount. These robots must transcend mere functional utility to engage humans in ways that feel natural, empathetic, and purposeful.

Effective human-robot interaction encompasses several critical dimensions. Empathy represents the robot's ability to recognize, understand, and appropriately respond to human emotions and social cues. Trust emerges from consistent, predictable, and reliable robot behavior that aligns with human expectations and social norms. Accessibility ensures that robots can interact meaningfully with users of varying abilities, ages, and technological backgrounds.

The need for such effective interaction stems from robotics applications in healthcare, education, elderly care, and therapeutic settings, where the quality of human-robot relationships directly impacts outcomes. These applications demand robots that can navigate complex social dynamics while maintaining their functional objectives, often relying on non-verbal communication through movement, gesture, and spatial behavior.

\section{The Tino Robot Project}
The Tino robot project at Politecnico di Milano investigates novel approaches to mobile robot social interaction through an interdisciplinary approach combining robotics engineering, human-computer interaction, and artificial intelligence research. Within the AIRLab robotics laboratory, the project focuses on natural movement, responsive behavior, and immersive control paradigms that bridge the gap between virtual and physical interaction spaces.

Tino is part of a larger research initiative where it acts as a digital medium for interaction between humans. In the envisioned scenario, one human interacts directly with the robot, while another controls the robot's movements remotely through a virtual reality interface. The two humans are unaware of each other's presence, with the robot serving as the communication medium. This setup enables the exploration of telepresence, empathy transfer, and mediated social interaction through robotic embodiment. The VR interface and virtual environment are being developed as part of a parallel thesis project, with this work focusing on the robot-side implementation that provides localization, orientation, and human pose data to the custom-built virtual space.

The robot's design philosophy emphasizes non-verbal and non-anthropomorphic features to build meaningful communication, convey emotions, and foster connections. By avoiding anthropomorphic design, Tino challenges conventional expectations of robotic form and demonstrates how purely physical movements can evoke empathy and emotional responses in human subjects. This approach enables exploration of movement as a communicative tool, independent of association with human anatomy.

The development of Tino V2 arose from the specific requirements of VR-based remote control and the limitations identified in the previous robot iteration. The legacy Tino system, while successful in demonstrating basic social interaction capabilities through direct local control, faced significant constraints when extended to real-time VR teleoperation. The original Raspberry Pi-based architecture with a Triskar omnidirectional base lacked the computational power necessary for real-time processing of VR commands, computer vision algorithms, and sophisticated sensor fusion required for remote operation.

The need for VR integration drove comprehensive system redesign focusing on enhanced computational capabilities, basic sensor integration, and low-latency communication systems. The transition to more powerful hardware platforms enables real-time artificial intelligence processing while maintaining the responsive, expressive movement capabilities essential for effective mediated human interaction.

\section{Project Objectives}

\subsection{Technical Objectives}
The Tino V2 project establishes several key technical objectives that address the limitations of the previous system while enabling advanced VR-mediated interaction capabilities. The primary goal is the development of a computational platform capable of real-time processing for VR teleoperation, enabling responsive and low-latency control essential for natural human-robot interaction through virtual reality interfaces.

Enhanced localization and navigation capabilities represent another critical objective, addressing the need for precise robot positioning and spatial awareness to provide accurate localization and orientation data to the VR system. This includes the development of basic sensor fusion techniques that combine multiple sensing modalities to achieve reliable robot pose estimation for transmission to the custom VR environment developed by a parallel thesis project.

The integration of advanced perception systems for human detection enables the creation of virtual human representations within the VR environment. These capabilities support the transfer of real-time human pose information to the VR operator, creating virtual avatars that represent the humans in the robot's actual surroundings and enable informed interaction decisions within the custom-built virtual space.

System reliability and performance improvements focus on developing a robust platform capable of sustained operation during extended interaction sessions. This includes mechanical enhancements, improved sensor integration, and basic sensor fallback behaviors that ensure consistent performance across diverse operational conditions.

\subsection{Research Objectives}
The research dimensions of the Tino V2 project contribute novel approaches to VR-mediated robotics and human-robot interaction. The investigation of immersive teleoperation paradigms explores how virtual reality interfaces can enable more natural and intuitive robot control, potentially improving the quality of mediated human interaction through robotic embodiment.

The development of adaptive movement systems designed specifically for VR control represents an innovative approach to robot teleoperation. This research investigates how complex robot behaviors can be decomposed into intuitive control primitives that feel natural when operated through virtual reality interfaces.

The study of real-time sensor data integration and environmental awareness in VR-controlled robots addresses the challenges of providing accurate robot localization and human pose data to the custom VR environment. This research explores techniques for reliable pose estimation and human detection that can be transmitted to the VR system developed by a parallel thesis project, enabling operators to make informed interaction decisions within the manually designed virtual space.

Advanced human-robot interaction paradigms emerge from the enhanced sensing and processing capabilities of the Tino V2 platform. The research investigates how VR-mediated control can preserve and enhance the expressive movement capabilities that enable emotional communication and empathy formation between humans and robots.

\section{Thesis Structure}
This thesis is organized to provide a comprehensive understanding of the Tino V2 development process, from foundational research through implementation and evaluation:

\begin{itemize}
    \item \textbf{Chapter 1: Introduction}: Presents the motivation for social robotics, the Tino project context, and research objectives.
    \item \textbf{Chapter 2: Background}: Covers the background research, technology survey, and detailed analysis of the legacy Tino system.
    \item \textbf{Chapter 3: Conceptual Work}: Details the overall system architecture, design decisions, and integration approach.
    \item \textbf{Chapter 4: Implementation}: Focuses on the technical implementation including hardware redesign, sensor fusion, human detection systems, and VR integration.
    \item \textbf{Chapter 5: Evaluation}: Provides evaluation results, testing procedures, and performance analysis.
    \item \textbf{Chapter 6: Conclusions}: Summarizes findings, contributions, and future research directions.
\end{itemize}