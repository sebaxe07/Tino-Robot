% A LaTeX template for MSc Thesis submissions to 
% Politecnico di Milano (PoliMi) - School of Industrial and Information Engineering
%
% S. Bonetti, A. Gruttadauria, G. Mescolini, A. Zingaro
% e-mail: template-tesi-ingind@polimi.it
%
% Last Revision: October 2021
%
% Copyright 2021 Politecnico di Milano, Italy. NC-BY

\documentclass{Configuration_Files/PoliMi3i_thesis}

%------------------------------------------------------------------------------
%	REQUIRED PACKAGES AND  CONFIGURATIONS
%------------------------------------------------------------------------------

% CONFIGURATIONS
\usepackage{parskip} % For paragraph layout
\usepackage{setspace} % For using single or double spacing
\usepackage{emptypage} % To insert empty pages
\usepackage{multicol} % To write in multiple columns (executive summary)
\setlength\columnsep{15pt} % Column separation in executive summary
\setlength\parindent{10pt} % Indentation
\raggedbottom  

% PACKAGES FOR TITLES
\usepackage{titlesec}
% \titlespacing{\section}{left spacing}{before spacing}{after spacing}
\titlespacing{\section}{0pt}{3.3ex}{2ex}
\titlespacing{\subsection}{0pt}{3.3ex}{1.65ex}
\titlespacing{\subsubsection}{0pt}{3.3ex}{1ex}
\usepackage{color}

% PACKAGES FOR LANGUAGE AND FONT
\usepackage[english]{babel} % The document is in English  
\usepackage[utf8]{inputenc} % UTF8 encoding
\usepackage[T1]{fontenc} % Font encoding
\usepackage[11pt]{moresize} % Big fonts

% PACKAGES FOR IMAGES
\usepackage{graphicx}
\usepackage{svg}
\usepackage{transparent} % Enables transparent images
\usepackage{eso-pic} % For the background picture on the title page
%\usepackage{subfig} % Numbered and caption subfigures using \subfloat.
\usepackage[export]{adjustbox}
\usepackage{tikz} % A package for high-quality hand-made figures.
\usetikzlibrary{}
\graphicspath{{./Images/}} % Directory of the images
\usepackage{caption} % Coloured captions
\usepackage{subcaption}
\usepackage{xcolor} % Coloured captions
\usepackage{amsthm,thmtools,xcolor} % Coloured "Theorem"
\usepackage{float}

% STANDARD MATH PACKAGES
\usepackage{amsmath}
\usepackage{amsthm}
\usepackage{amssymb}
\usepackage{amsfonts}
\usepackage{bm}
\usepackage{cancel}
\usepackage[overload]{empheq} % For braced-style systems of equations.
\usepackage{fix-cm} % To override original LaTeX restrictions on sizes

% PACKAGES FOR TABLES
\usepackage{tabularx}
\usepackage{longtable} % Tables that can span several pages
\usepackage{colortbl}

% PACKAGES FOR ALGORITHMS (PSEUDO-CODE)
\usepackage{algorithm}
\usepackage{algorithmic}

% PACKAGES FOR REFERENCES & BIBLIOGRAPHY
\usepackage[colorlinks=true,linkcolor=black,anchorcolor=black,citecolor=black,filecolor=black,menucolor=black,runcolor=black,urlcolor=black]{hyperref} % Adds clickable links at references
\usepackage{cleveref}
\usepackage[square, numbers, sort&compress]{natbib} % Square brackets, citing references with numbers, citations sorted by appearance in the text and compressed
\bibliographystyle{abbrvnat} % You may use a different style adapted to your field

% OTHER PACKAGES
\usepackage{pdfpages} % To include a pdf file
\usepackage{afterpage}
\usepackage{lipsum} % DUMMY PACKAGE
\usepackage{fancyhdr} % For the headers
\usepackage{fancyvrb}
\usepackage[acronym]{glossaries}
\usepackage{enumitem} 
\fancyhf{}

\usepackage{multirow}
\usepackage{booktabs}
\usepackage{placeins}


% Input of configuration file. Do not change config.tex file unless you really know what you are doing. 
% Define blue color typical of polimi
\definecolor{bluepoli}{cmyk}{0.4,0.1,0,0.4}
\definecolor{Green}{RGB}{30, 170, 0}

% Custom theorem environments
\declaretheoremstyle[
  headfont=\color{bluepoli}\normalfont\bfseries,
  bodyfont=\color{black}\normalfont\itshape,
]{colored}

% Set-up caption colors
%\captionsetup[figure]{labelfont={color=bluepoli}} % Set colour of the captions
%\captionsetup[table]{labelfont={color=bluepoli}} % Set colour of the captions
%\captionsetup[algorithm]{labelfont={color=bluepoli}} % Set colour of the captions
\captionsetup{font=bf}

\theoremstyle{colored}
\newtheorem{theorem}{Theorem}[chapter]
\newtheorem{proposition}{Proposition}[chapter]
\newtheorem{definition}{Definition}[chapter]

% Enhances the features of the standard "table" and "tabular" environments.
\newcommand\T{\rule{0pt}{2.6ex}}
\newcommand\B{\rule[-1.2ex]{0pt}{0pt}}

% Pseudo-code algorithm descriptions.
\newcounter{algsubstate}
\renewcommand{\thealgsubstate}{\alph{algsubstate}}
\newenvironment{algsubstates}
  {\setcounter{algsubstate}{0}%
   \renewcommand{\STATE}{%
     \stepcounter{algsubstate}%
     \Statex {\small\thealgsubstate:}\space}}
  {}

% New font size
\newcommand\numfontsize{\@setfontsize\Huge{200}{60}}

% Title format: chapter
\titleformat{\chapter}[hang]{
\fontsize{50}{20}\selectfont\bfseries\filright}{\textcolor{bluepoli} \thechapter\hsp\hspace{2mm}\textcolor{bluepoli}{|   }\hsp}{0pt}{\huge\bfseries \textcolor{bluepoli}
}

% Title format: section
\titleformat{\section}
{\normalfont\Large\bfseries}
{\thesection.}{1em}{}
%{\color{bluepoli}\normalfont\Large\bfseries}
%{\color{bluepoli}\thesection.}{1em}{}

% Title format: subsection
\titleformat{\subsection}
{\normalfont\large\bfseries}
{\thesubsection.}{1em}{}
%{\color{bluepoli}\normalfont\large\bfseries}
%{\color{bluepoli}\thesubsection.}{1em}{}

% Title format: subsubsection
\titleformat{\subsubsection}
{\normalfont\bfseries}
{\thesubsubsection.}{1em}{}
%{\color{bluepoli}\normalfont\large\bfseries}
%{\color{bluepoli}\thesubsubsection.}{1em}{}

% Shortening for setting no horizontal-spacing
\newcommand{\hsp}{\hspace{0pt}}

\makeatletter
% Renewcommand: cleardoublepage including the background pic
\renewcommand*\cleardoublepage{%
  \clearpage\if@twoside\ifodd\c@page\else
  \null
  \AddToShipoutPicture*{\BackgroundPic}
  \thispagestyle{empty}%
  \newpage
  \if@twocolumn\hbox{}\newpage\fi\fi\fi}
\makeatother

%For correctly numbering algorithms
\numberwithin{algorithm}{chapter}

%----------------------------------------------------------------------------
%	NEW COMMANDS DEFINED
%----------------------------------------------------------------------------

% EXAMPLES OF NEW COMMANDS
\newcommand{\bea}{\begin{eqnarray}} % Shortcut for equation arrays
\newcommand{\eea}{\end{eqnarray}}
\newcommand{\e}[1]{\times 10^{#1}}  % Powers of 10 notation

%----------------------------------------------------------------------------
%	ADD YOUR PACKAGES (be careful of package interaction)
%----------------------------------------------------------------------------

%----------------------------------------------------------------------------
%	ADD YOUR DEFINITIONS AND COMMANDS (be careful of existing commands)
%----------------------------------------------------------------------------

\makeglossaries

\newacronym{acronym}{ACR}{Acronym}

%----------------------------------------------------------------------------
%	BEGIN OF YOUR DOCUMENT
%----------------------------------------------------------------------------

\begin{document}

\fancypagestyle{plain}{%
\fancyhf{} % Clear all header and footer fields
\fancyhead[RO,RE]{\thepage} %RO=right odd, RE=right even
\renewcommand{\headrulewidth}{0pt}
\renewcommand{\footrulewidth}{0pt}}

%----------------------------------------------------------------------------
%	TITLE PAGE
%----------------------------------------------------------------------------

\pagestyle{empty} % No page numbers
\frontmatter % Use roman page numbering style (i, ii, iii, iv...) for the preamble pages

\puttitle{
	title={Tino V2, the revenge of the maze (Temporal Title)}, % Title of the thesis
	name=Sebastian Enrique Perea Lopez, % Author Name and Surname
	course=Computer Science and Engineering \\Ingegneria Informatica, % Study Programme (in Italian)
	ID  =  10986638,  % Student ID number (numero di matricola)
	advisor= Prof. Andrea Bonarini, % Supervisor name
	coadvisor={Federico Espositi}, % Co-Supervisor name, remove this line if there is none
	academicyear={2024-25},  % Academic Year
} % These info will be put into your Title page 

%----------------------------------------------------------------------------
%	PREAMBLE PAGES: ABSTRACT (inglese e italiano), EXECUTIVE SUMMARY
%----------------------------------------------------------------------------

\pagebreak
\pagestyle{empty}
\hspace{0pt}
\vfill
\textit{Dedicated to my family.}
\vfill
\hspace{0pt}
\pagebreak

\startpreamble
\setcounter{page}{1} % Set page counter to 1

% ABSTRACT IN ENGLISH
\chapter*{Abstract} 
Here goes the abstract.
\\
\\
\textbf{Keywords:} key, words, go, here% Keywords

% ABSTRACT IN ITALIAN
\chapter*{Abstract in lingua italiana}
Qui va inserito l'abstract in italiano.
\\
\\
\textbf{Parole chiave:} qui, vanno, le, parole, chiave% Keywords (italian)

%----------------------------------------------------------------------------
%	LIST OF CONTENTS/FIGURES/TABLES/SYMBOLS
%----------------------------------------------------------------------------

% TABLE OF CONTENTS
\thispagestyle{empty}
\tableofcontents % Table of contents 
\thispagestyle{empty}
\cleardoublepage

%-------------------------------------------------------------------------
%	THESIS MAIN TEXT
%-------------------------------------------------------------------------
% In the main text of your thesis you can write the chapters in two different ways:
%
%(1) As presented in this template you can write:
%    \chapter{Title of the chapter}
%    *body of the chapter*
%
%(2) You can write your chapter in a separated .tex file and then include it in the main file with the following command:
%    \chapter{Title of the chapter}
%    \input{chapter_file.tex}
%
% Especially for long thesis, we recommend you the second option.


\addtocontents{toc}{\vspace{2em}} % Add a gap in the Contents, for aesthetics
\mainmatter % Begin numeric (1,2,3...) page numbering


% Chapter 1: Background and Objectives
\chapter{Introduction and Project Overview}
\section{Background and Context}
This section will provide a comprehensive overview of the social robotics field and human-robot interaction research, establishing the foundation for understanding the Tino robot project's significance. The discussion will begin with a brief history of social robotics development and its evolution in academic and commercial applications. Following this, the section will introduce the Tino robot project as developed at Politecnico di Milano, detailing its role in advancing human-robot interaction research within the academic community. An overview of the original Tino robot will be presented, including its design philosophy, intended applications, and practical use cases in social interaction scenarios. Finally, the research context will be established within the AIRLab robotics laboratory, highlighting the interdisciplinary approach and research objectives that drive the Tino project development.

\section{Legacy Tino System Overview}
This section will present a detailed technical analysis of the original Tino robot architecture, providing the foundation for understanding the improvements made in Tino V2. The discussion will begin with a comprehensive description of the legacy system's hardware and software components, focusing on the Raspberry Pi-based control system and its inherent limitations in processing power and real-time performance. The original omnidirectional movement system utilizing the Triksta base will be examined, including its mechanical design, control algorithms, and operational characteristics. The basic sensor setup and capabilities of the legacy system will be detailed, covering the available sensing modalities and their integration into the robot's behavior system. Previous applications and use cases will be documented, demonstrating the robot's practical deployment in social interaction scenarios. Finally, the section will identify specific limitations and areas for improvement that motivated the development of Tino V2, including hardware constraints, software architecture issues, and performance bottlenecks.

\section{Project Motivation and Objectives}
This section will articulate the driving forces behind the Tino V2 development project and establish clear objectives for the research and development effort. The discussion will begin by identifying key limitations of the legacy system that necessitated a comprehensive redesign, including inadequate computational resources, limited sensor capabilities, and reliability issues. The need for improved localization and navigation capabilities will be established, highlighting the requirements for precise positioning in complex indoor environments and robust SLAM implementation. Requirements for VR integration and immersive interaction will be detailed, explaining how these capabilities extend the robot's potential applications and research value. Hardware reliability and performance enhancement needs will be discussed, focusing on the mechanical improvements required for sustained operation and user interaction. The section will conclude with an overview of research goals for advanced human-robot interaction, establishing the academic and practical contributions expected from the Tino V2 project.

\subsection{Technical Objectives}
This subsection will detail the specific technical goals that guide the Tino V2 development process. The migration from the Raspberry Pi platform to the NVIDIA Orin Nano will be explained, including the performance benefits and expanded capabilities this transition enables. The implementation of robust SLAM (Simultaneous Localization and Mapping) and sensor fusion techniques will be outlined, describing how these technologies address the localization challenges identified in the legacy system. The development of real-time human detection and pose estimation capabilities will be presented, explaining the integration of computer vision and machine learning technologies for enhanced human-robot interaction. The creation of VR-compatible control and interaction systems will be detailed, describing the software architecture and communication protocols required for seamless virtual reality integration. Finally, the improvement of mechanical reliability and performance will be discussed, covering hardware redesign efforts and component upgrades that enhance the robot's operational capabilities.

\subsection{Research Objectives}
This subsection will outline the research contributions and academic goals of the Tino V2 project. The investigation of hybrid localization approaches combining SLAM with Ultra-Wideband (UWB) positioning will be presented as a novel contribution to mobile robotics research. The development of atomic movement systems specifically designed for VR control will be detailed, explaining how this approach enables more natural and intuitive human-robot interaction paradigms. The exploration of advanced human-robot interaction paradigms will be discussed, focusing on how the enhanced sensing and control capabilities enable new forms of social interaction and user engagement. The study of real-time pose detection integration with robot behavior will be presented, demonstrating how human posture and gesture recognition can be incorporated into responsive robot behaviors for more natural interaction experiences.

\section{System Requirements and Constraints}
This section will establish the technical specifications and operational constraints that define the Tino V2 system design. Performance requirements for real-time operation will be detailed, including computational load specifications, response time requirements, and throughput expectations for various system components. Accuracy requirements for localization and human tracking will be specified, establishing the precision standards necessary for effective navigation and interaction capabilities. VR integration constraints and specifications will be outlined, including latency requirements, data exchange protocols, and synchronization standards required for seamless virtual reality operation. Physical constraints will be discussed, covering weight limitations, size restrictions, and power consumption requirements that influence the hardware design and component selection. Finally, reliability and robustness requirements for social interaction will be established, defining the operational standards necessary for safe and effective human-robot interaction in various environments.

\section{Thesis Structure and Contributions}
This section will provide readers with a clear roadmap of the thesis organization and highlight the key contributions of the research. An overview of the thesis organization and chapter flow will be presented, explaining how each chapter builds upon previous work and contributes to the overall narrative of the Tino V2 development project. The summary of main technical contributions will detail the engineering innovations and system improvements achieved in the project, including hardware upgrades, software architecture enhancements, and integration achievements. The summary of research contributions will highlight the academic value of the work, including novel approaches to mobile robot localization, human-robot interaction paradigms, and VR integration techniques. Key innovations and novel approaches introduced in the project will be emphasized, demonstrating how the work advances the state of the art in social robotics and mobile robot systems. Finally, the expected impact and applications of the developed system will be discussed, explaining how the Tino V2 platform enables new research opportunities and practical applications in human-robot interaction, education, and social robotics research.


% Chapter 2: State of the Art and Technology Selection
\chapter{Technology Research and Selection}
\section{Localization Technologies Review}
This section will provide a comprehensive analysis of available localization technologies for mobile robotics applications, establishing the foundation for the technology selection decisions made in the Tino V2 project. The review will begin with an examination of Visual Odometry (VO) approaches, including both monocular and stereo camera implementations, analyzing their strengths in feature-rich environments and limitations in scenarios with poor lighting or repetitive textures. The discussion will cover ORB-SLAM3 capabilities for RGB-D sensor integration, SVO (Semi-direct Visual Odometry) advantages for fisheye and wide field-of-view cameras, and the computational requirements associated with each approach. Ultra-Wideband (UWB) positioning technology will be evaluated for its centimeter-level accuracy potential, low power consumption characteristics, and infrastructure requirements including anchor placement and calibration procedures. The analysis will address fabric penetration capabilities crucial for Tino's soft structure, Non-Line-of-Sight (NLOS) mitigation strategies, and orientation estimation challenges inherent in UWB systems. Finally, IMU and wheel encoder fusion approaches will be examined, including Extended Kalman Filter (EKF) implementations, drift accumulation issues, and performance limitations on uneven surfaces and with impulse-based movement patterns.

\section{Human Detection Technologies Comparison}
This section will systematically evaluate available human detection and tracking technologies suitable for social robotics applications. RGB-D camera solutions will be analyzed first, examining Intel RealSense and similar depth-sensing cameras for their simultaneous color and depth data capabilities, skeleton tracking potential using OpenPose and MediaPipe frameworks, and integration challenges related to physical mounting and visibility constraints within Tino's fabric structure. Thermal imaging technology will be evaluated for its potential fabric penetration capabilities, performance in low-light conditions, and limitations in providing contextual information beyond heat signatures. The discussion will cover fusion strategies between thermal sensors and other sensing modalities to achieve comprehensive human detection capabilities. LiDAR technology will be assessed for its high-resolution 3D mapping capabilities, though acknowledging its impracticality for Tino's soft structure due to vibration sensitivity and cost considerations. Machine Learning-enhanced 2D camera approaches will be examined, including modern architectures such as YOLOv8 and EfficientNet for real-time detection, monocular depth estimation networks like MiDaS and LeReS, and the computational requirements for real-time processing on embedded platforms.

\section{SLAM Systems Evaluation}
This section will present a detailed technical evaluation of Simultaneous Localization and Mapping (SLAM) systems considered for the Tino V2 implementation. ORB-SLAM3 will be examined first, analyzing its support for monocular, stereo, and RGB-D sensor configurations, robustness in dynamic environments through advanced feature matching, and computational requirements that necessitate GPU optimization for real-time performance. The evaluation will cover map persistence capabilities, loop closure detection mechanisms, and integration challenges encountered during development. SVO (Semi-direct Visual Odometry) will be analyzed for its compatibility with fisheye and catadioptric cameras, reduced computational footprint compared to feature-based methods, and performance limitations in textureless environments. The discussion will include practical implementation challenges and compilation issues encountered on ARM64 architectures. RTABMap (Real-Time Appearance-Based Mapping) will be evaluated as the selected SLAM solution, examining its multi-session mapping capabilities, robust map saving and loading functionality, and superior relocalization performance. The analysis will cover integration with the Oak-D Pro camera through the DepthAI library, ROS2 compatibility, and performance characteristics that made it the optimal choice for the Tino V2 platform.

\section{Technology Selection Rationale}
This section will present the systematic decision-making process that led to the final technology stack selection for Tino V2. The evaluation criteria will be established first, including accuracy requirements for social interaction scenarios, computational efficiency constraints of embedded platforms, integration complexity with existing systems, and reliability requirements for sustained operation. For localization technology selection, the analysis will explain why a hybrid approach combining RTABMap SLAM with UWB positioning was chosen over single-technology solutions. The decision process will cover RTABMap's superior map persistence and relocalization capabilities compared to ORB-SLAM3 and SVO, UWB technology's potential for absolute positioning to address SLAM drift issues, and the complementary nature of visual orientation data from SLAM with precise positioning from UWB systems. For human detection technology selection, the rationale for choosing YOLOv11 with TensorRT optimization will be presented, including real-time performance capabilities, accuracy in detecting multiple humans simultaneously, and successful integration with stereo depth data for 3D positioning. The section will address why this approach was preferred over thermal imaging or pure RGB-D solutions, considering Tino's specific operational requirements and physical constraints.

\section{Hybrid Approach Justification}
This section will provide detailed justification for the hybrid localization approach that combines multiple sensing modalities to achieve superior performance compared to individual technologies. The limitations of SLAM-only approaches will be discussed first, including drift accumulation over extended operation periods, relocalization failures in feature-poor environments, and map corruption issues that can compromise long-term operation. Specific examples from the development process will illustrate scenarios where RTABMap exhibited positioning drift up to 1.2 meters, necessitating manual intervention or robot rotation to achieve relocalization. The complementary capabilities of UWB positioning will be explained, demonstrating how centimeter-level absolute positioning addresses SLAM drift issues while maintaining the rich environmental understanding provided by visual SLAM systems. The sensor fusion strategy will be detailed, explaining how RTABMap provides reliable orientation information and environmental mapping while UWB delivers precise global positioning, creating a robust localization system that leverages the strengths of both technologies. Performance comparisons will be presented showing the improved accuracy and reliability achieved through the hybrid approach compared to individual sensor modalities.

\section{Selected Technology Stack for Tino V2}
This section will present the final integrated technology stack selected for the Tino V2 platform, providing a comprehensive overview of how individual components work together to achieve the project objectives. The hardware platform selection will be detailed first, explaining the migration from Raspberry Pi to NVIDIA Orin Nano and the performance benefits this transition enables for real-time SLAM processing, computer vision algorithms, and multi-modal sensor fusion. The Oak-D Pro camera selection will be justified for its stereo depth capabilities, DepthAI library integration, and compatibility with RTABMap SLAM implementation. The localization system architecture will be presented, describing the integration of RTABMap SLAM for visual odometry and mapping with UWB positioning for absolute coordinate reference, including the sensor fusion algorithms that combine these data streams. The human detection pipeline will be detailed, covering YOLOv11 implementation with TensorRT optimization, stereo depth integration for 3D human positioning, and real-time skeleton tracking with 17 key body joints. The software architecture selection will explain the migration to ROS2 for improved modularity, the node-based system design that enables independent development and testing of subsystems, and the communication protocols that facilitate integration with VR systems and external applications.


% Chapter 3: System Architecture and ROS2 Migration
\chapter{System Architecture and Software Migration}
\section{Legacy System Analysis and Limitations}
This section will provide a comprehensive analysis of the original Tino robot system architecture, establishing the technical foundation that necessitated the migration to Tino V2. The legacy system evaluation will begin with a detailed examination of the Raspberry Pi-based control architecture, including its computational limitations for real-time processing, memory constraints that restricted simultaneous operation of multiple sensing modalities, and processing bottlenecks that prevented implementation of advanced computer vision algorithms. The original software architecture will be analyzed, covering the monolithic Python script design that combined all robot functionalities into single-threaded execution, leading to poor modularity, difficult debugging procedures, and limited scalability for adding new features. The communication system limitations will be discussed, including the lack of standardized message protocols, direct serial communication bottlenecks with Arduino subsystems, and absence of distributed processing capabilities. The sensor integration challenges will be examined, particularly the limited bandwidth for camera data processing, inadequate computational resources for SLAM implementation, and difficulties in achieving synchronized multi-sensor operation. Finally, the section will address reliability and maintenance issues, including system crashes due to resource exhaustion, difficulties in isolating and fixing component failures, and the challenges of remote debugging and system monitoring.

\section{ROS2 Architecture Design and Implementation}
This section will detail the comprehensive ROS2 architecture designed for Tino V2, explaining how the Robot Operating System 2 framework addresses the limitations identified in the legacy system. The ROS2 framework selection rationale will be presented first, covering its advantages in distributed computing, real-time performance capabilities, improved inter-process communication through DDS (Data Distribution Service), and enhanced security features compared to ROS1. The system architecture design principles will be explained, including the modular node-based approach that enables independent development and testing of subsystems, standardized message interfaces that facilitate component integration, and distributed processing capabilities that leverage the Orin Nano's enhanced computational resources. The communication infrastructure will be detailed, covering the publish-subscribe messaging paradigm, quality of service (QoS) policies for reliable data transmission, and the discovery mechanisms that enable automatic node detection and connection. The real-time considerations will be addressed, including deterministic message delivery, priority-based scheduling, and resource allocation strategies that ensure consistent performance for time-critical operations. The development and deployment advantages will be discussed, covering the improved debugging capabilities through ROS2 tools, enhanced logging and monitoring systems, and the simplified integration with external systems and simulation environments.

\section{Node Structure and Functionality}
This section will provide detailed descriptions of the individual ROS2 nodes that compose the Tino V2 system, explaining their specific responsibilities and interactions within the overall architecture. The gamepad\_node.py will be examined first, detailing its role in handling Xbox controller input with proper D-input to X-input conversion for Jetson compatibility, the implementation of the pulse-based command system that generates discrete movement commands for VR integration, and the enhanced command processing and error reporting mechanisms that improve control system robustness. The hardware\_interface\_node.py will be analyzed, covering its management of serial communication with all three Arduino systems (head, base, leg), the implementation of proper device symlinks for consistent Arduino connections, and the comprehensive debugging logs and status monitoring that enable effective system maintenance. The robot\_controller\_node.py functionality will be detailed, explaining its role as the central coordination node that manages all robot behaviors and movement commands, the implementation of the atomic movement system with 4-state leg and base control, and the synchronization mechanisms that ensure coordinated robot actions. The vr\_interface\_node.py will be examined, covering its handling of VR system integration and data exchange for Unity communication, the implementation of bidirectional audio communication systems, and the data recording and extraction capabilities that support VR system development and testing.

\section{Communication Protocols and Message Design}
This section will detail the communication infrastructure and message protocols developed for the Tino V2 system, ensuring reliable and efficient data exchange between all system components. The inter-node communication design will be explained first, covering the topic-based messaging system that enables decoupled component interaction, the service-call mechanisms for synchronous operations requiring immediate responses, and the action-based communication for long-running tasks with progress feedback. The custom message definitions will be detailed, including pose and localization messages that carry 3D position and orientation data with timestamp information, human detection messages that contain skeleton joint positions and confidence scores, and VR interface messages that facilitate bidirectional communication with Unity systems. The Arduino communication protocols will be examined, covering the enhanced serial communication protocols that provide reliable command transmission and status feedback, the implementation of command acknowledgment systems that ensure successful command execution, and the error handling mechanisms that detect and recover from communication failures. The data synchronization strategies will be discussed, including timestamp-based alignment of multi-sensor data, buffering mechanisms that handle varying data rates from different sensors, and the quality of service configurations that prioritize critical data streams while ensuring overall system stability.

\section{Integration with External Systems}
This section will explain how the ROS2 architecture facilitates integration with external systems and development tools, enhancing the overall capability and research value of the Tino V2 platform. The VR system integration will be detailed first, covering the ROS-TCP-Endpoint implementation that provides a communication bridge between ROS2 and Unity, the enhanced error handling mechanisms that ensure robust VR communication, and the data recording systems that capture comprehensive robot state information for VR development and analysis. The monitoring and debugging capabilities will be discussed, covering the comprehensive logging systems that track all node activities and system events, the diagnostic tools that provide real-time system health monitoring, and the remote access capabilities that enable development and troubleshooting via VNC connections. The extensibility features will be addressed, explaining how the modular architecture enables easy addition of new sensors and capabilities, the standardized interfaces that facilitate integration of third-party components, and the configuration management systems that allow dynamic parameter adjustment without system restart.

\section{Migration Benefits and System Improvements}
This section will quantify and analyze the improvements achieved through the migration from the legacy Raspberry Pi system to the ROS2-based Orin Nano architecture. The performance improvements will be detailed first, including the computational performance gains that enable real-time SLAM processing, computer vision algorithms, and multi-sensor fusion, the memory management improvements that allow simultaneous operation of multiple demanding processes, and the reduced latency in sensor data processing and command execution. The reliability enhancements will be examined, covering the improved system stability through modular architecture that isolates component failures, the enhanced error handling and recovery mechanisms that maintain system operation during partial failures, and the diagnostic capabilities that enable proactive maintenance and issue identification. The development efficiency improvements will be discussed, including the reduced development time for new features through standardized interfaces and modular design, the improved debugging capabilities that enable rapid issue identification and resolution, and the enhanced testing procedures that allow independent validation of system components. The operational advantages will be addressed, covering the simplified system startup and shutdown procedures, the improved remote monitoring and control capabilities, and the enhanced data logging and analysis tools that support research activities. Finally, the scalability benefits will be presented, explaining how the new architecture supports future expansion with additional sensors and capabilities, enables distributed processing across multiple computing platforms, and facilitates integration with cloud-based services and external research systems.


% Chapter 4: SLAM Implementation and Localization
\chapter{SLAM Implementation and Sensor Fusion}
\section{RTABMap Integration with Oak-D Pro Camera}
This section will detail the comprehensive implementation of RTABMap (Real-Time Appearance-Based Mapping) SLAM system with the Oak-D Pro camera, establishing the foundation for Tino V2's localization capabilities. The Oak-D Pro camera integration will be explained first, covering the DepthAI library implementation that provides seamless access to stereo depth data, the ROS2 wrapper configuration that publishes synchronized color and depth image streams, and the camera calibration procedures that ensure accurate depth estimation and feature detection. The RTABMap configuration process will be detailed, including the parameter optimization for indoor environments, memory management settings that enable long-term operation without performance degradation, and the feature detection and matching algorithms that provide robust visual odometry in dynamic social interaction scenarios. The stereo vision implementation will be examined, covering how the Oak-D Pro's dual camera system provides depth information for feature triangulation, the baseline calibration that ensures accurate 3D point cloud generation, and the integration with RTABMap's visual-inertial odometry algorithms. The real-time performance optimization will be discussed, including the computational load balancing between feature extraction and map building processes, the memory allocation strategies that prevent system crashes during extended mapping sessions, and the parameter tuning that achieves optimal performance on the Orin Nano platform.

\section{SLAM Mapping and Localization Modes}
This section will explain the dual operational modes implemented in the Tino V2 system, enabling both map creation and autonomous navigation capabilities. The mapping mode implementation will be detailed first, covering the launch file configuration (rtab\_mapping.launch.py) that initializes all necessary nodes for map creation, the real-time visualization capabilities that allow monitoring of map building progress, and the landmark placement strategies that improve map quality and relocalization reliability. The map building process will be examined, including the loop closure detection mechanisms that ensure map consistency, the keyframe selection algorithms that optimize memory usage while maintaining map quality, and the feature database management that enables efficient storage and retrieval of visual landmarks. The localization mode implementation will be explained, covering the launch file configuration (rtab\_localization.launch.py) that loads existing maps and initializes positioning systems, the relocalization algorithms that enable the robot to determine its position within a previously created map, and the continuous tracking mechanisms that maintain accurate positioning during navigation. The mode switching procedures will be detailed, including the map saving and loading protocols that ensure data persistence, the system state management that enables seamless transitions between mapping and localization modes, and the parameter configuration changes required for optimal performance in each operational mode.

\section{Initial SLAM-Only System Limitations and Drift Issues}
This section will present a comprehensive analysis of the limitations discovered during initial testing of the SLAM-only localization approach, providing the technical justification for implementing the hybrid sensor fusion system. The drift accumulation problems will be documented first, including specific measurement data showing position errors of up to 1.2 meters during extended operation, the systematic analysis of error sources including visual odometry drift and map inconsistencies, and the environmental factors that contribute to localization degradation such as lighting changes and repetitive textures. The relocalization challenges will be examined, covering scenarios where RTABMap failed to correctly determine robot position within existing maps, the feature-poor environments that caused tracking failures, and the manual intervention requirements (robot rotation) needed to achieve successful relocalization. The performance testing methodology will be detailed, including the four-position testing protocol implemented to quantify localization accuracy, the repeated measurement procedures that revealed consistency issues, and the statistical analysis of position errors that demonstrated the need for absolute positioning reference. The specific failure modes will be analyzed, covering map corruption events that required complete map reconstruction, tracking loss scenarios that occurred near walls or in areas with insufficient visual features, and the computational resource limitations that affected real-time performance during intensive mapping operations.

\section{UWB Positioning System Implementation}
This section will detail the Ultra-Wideband (UWB) positioning system integration that provides absolute positioning reference to complement the visual SLAM system. The UWB hardware implementation will be explained first, covering the anchor placement strategy that ensures optimal coverage of the operating environment, the calibration procedures that establish accurate position references, and the tag integration with the Tino robot platform that minimizes interference with other systems. The positioning algorithm implementation will be detailed, including the multilateration techniques that calculate 3D position from multiple anchor measurements, the Non-Line-of-Sight (NLOS) mitigation strategies that improve accuracy in complex indoor environments, and the filtering algorithms that reduce measurement noise and provide stable position estimates. The coordinate system alignment will be examined, covering the transformation procedures that align UWB coordinates with the SLAM coordinate frame, the map offset implementation that enables rotational correction of SLAM maps, and the calibration protocols that ensure consistent positioning across different operational sessions. The real-time performance characteristics will be analyzed, including the update rates achieved by the UWB system, the latency measurements that demonstrate suitability for real-time control applications, and the accuracy evaluation that shows centimeter-level positioning precision under optimal conditions.

\section{Sensor Fusion Between RTABMap Orientation and UWB Positioning}
This section will present the sophisticated sensor fusion approach that combines the complementary strengths of visual SLAM and UWB positioning to achieve superior localization performance. The fusion algorithm design will be explained first, covering the Extended Kalman Filter (EKF) implementation that optimally combines position and orientation data from multiple sources, the state estimation procedures that maintain accurate robot pose estimates, and the uncertainty quantification methods that provide confidence measures for navigation decisions. The data synchronization implementation will be detailed, including the timestamp alignment procedures that ensure temporal consistency between sensor measurements, the interpolation algorithms that handle different update rates from SLAM and UWB systems, and the buffering mechanisms that maintain data integrity during temporary sensor outages. The orientation and position separation strategy will be examined, covering how RTABMap provides reliable orientation information while UWB delivers absolute position data, the coordinate transformation procedures that maintain consistency between different sensor coordinate frames, and the validation algorithms that detect and reject outlier measurements. The adaptive fusion parameters will be discussed, including the dynamic weighting strategies that adjust fusion coefficients based on sensor reliability, the fault detection mechanisms that identify sensor malfunctions or degraded performance, and the graceful degradation procedures that maintain operation when individual sensors fail.

\section{Final Hybrid Localization System Performance}
This section will present comprehensive performance evaluation of the final hybrid localization system, demonstrating the improvements achieved through sensor fusion compared to individual sensing modalities. The accuracy evaluation will be detailed first, including quantitative measurements comparing SLAM-only, UWB-only, and hybrid system performance, statistical analysis of position errors showing significant improvement in the fused system, and repeatability testing that demonstrates consistent performance across multiple operational sessions. The specific performance data will be presented, covering the four-position testing results that show centimeter-level accuracy with the hybrid system, the position consistency measurements that demonstrate reduced drift compared to SLAM-only operation, and the orientation accuracy evaluation that validates the continued use of visual odometry for heading estimation. The operational reliability improvements will be examined, including the reduced need for manual relocalization procedures, the improved performance in challenging environments with poor visual features, and the enhanced system stability during extended autonomous operation. The computational performance analysis will be discussed, covering the processing load distribution between SLAM and UWB systems, the real-time performance characteristics that meet the requirements for responsive robot control, and the memory usage optimization that enables continuous operation without system degradation. Finally, the system robustness evaluation will be presented, including fault tolerance testing that demonstrates graceful degradation when individual sensors fail, environmental adaptability testing that shows consistent performance across different lighting and structural conditions, and long-term stability evaluation that validates the system's suitability for extended autonomous operation in social robotics applications.


% Chapter 5: Hardware Redesign and Mechanical Improvements
\chapter{Hardware Redesign and Mechanical Improvements}
\section{Kinematic Base Upgrade from Omnidirectional to Differential Drive}
This section will detail the comprehensive redesign of Tino's mobility system, transitioning from the problematic omnidirectional Triksta base to a robust differential drive architecture. The limitations of the original omnidirectional system will be analyzed first, covering the mechanical failures experienced with the omniwheel rollers that became squared due to Tino's 20kg weight, the dragging issues with the rear wheel that occurred during forward and turning movements, and the unreliable motor performance under the sustained loads required for social robot operation. The differential drive design rationale will be explained, including the simplified kinematics that eliminate the complexity of omnidirectional control while maintaining adequate maneuverability for social interaction scenarios, the improved weight distribution that reduces stress on individual components, and the enhanced reliability achieved through proven mechanical design principles. The mechanical implementation will be detailed, covering the construction of the T-structure using aluminum Item profiles that provide a dynamic and adjustable framework, the motor mounting system modifications required to accommodate the new differential drive configuration, and the wheel positioning optimization that achieves proper balance and traction for the robot's operational requirements. The control system adaptation will be examined, including the implementation of custom PID (Proportional-Integral-Derivative) controllers specifically designed for differential drive kinematics, the motor driver upgrade to the more powerful MDD10A units that can handle increased loads, and the command interface modifications that maintain compatibility with existing movement control systems while improving performance and reliability.

\section{Power Supply System Redesign for Orin Nano}
This section will present the comprehensive power system redesign required to support the NVIDIA Orin Nano platform and associated high-performance components. The power requirements analysis will be detailed first, covering the Orin Nano's 19V DC input requirement and power consumption characteristics that reach up to 2A during maximum computational load, the additional power needs for the Oak-D Pro camera and onboard router systems, and the total system power budget that necessitated complete redesign of the legacy Raspberry Pi power architecture. The DC-DC converter implementation will be explained, including the selection and testing of the Oumefar 12V to 19V step-up converter that provides stable power delivery, the power efficiency analysis that demonstrates optimal battery utilization, and the thermal management considerations that ensure reliable operation under sustained loads. The battery system optimization will be examined, covering the consolidation from four separate battery systems to three integrated power sources, the 5200mAh 80C 11.1V 57.72Wh battery specification that provides approximately 1.37 hours of operation at maximum load, and the realistic operational time estimates of 2-3 hours under typical social interaction scenarios. The cable harness redesign will be detailed, including the removal of legacy USB-A and USB-C connections that were used for Raspberry Pi power delivery, the implementation of proper 12V input distribution and 19V DC jack connectivity, and the integration of the 12V to 5V converter that powers the onboard router and camera systems independently, providing flexibility for future system expansions and reducing the computational load on the main platform.

\section{Stewart Platform Head Mechanism Improvements}
This section will document the iterative design improvements made to Tino's Stewart platform head mechanism to address reliability issues and enhance performance under operational loads. The original system limitations will be analyzed first, covering the servo axis misalignment problems that created excessive stress on servo motors during head movements, the structural flex issues in the connecting arms that caused mechanical instability and reduced precision, and the repeated arm failures that occurred due to inadequate load distribution and material selection. The first design iteration will be detailed, including the servo axis alignment improvement that redirected forces through the head structure rather than the servo mechanisms, the 3D printed PLA arm replacement with enhanced geometry for improved load distribution, and the initial performance evaluation that showed reduced servo stress but continued structural flex issues. The final design implementation will be examined, covering the adoption of rod end (heim joints) on both ends of each Stewart platform arm to eliminate binding and allow free rotation, the combination of 3D printed components with metal heim joints that provides optimal balance between cost and performance, and the mechanical trade-offs including acceptable head wobble during stationary periods that may actually enhance the robot's expressive capabilities. The performance validation will be discussed, including load testing that demonstrates improved reliability under operational conditions, the movement precision evaluation that shows maintained accuracy despite the mechanical improvements, and the longevity testing that validates the enhanced design's suitability for extended social interaction scenarios.

\section{Camera Integration and Mounting Solutions}
This section will detail the comprehensive camera integration system developed to address the unique challenges of mounting sophisticated sensing equipment within Tino's soft fabric structure. The mounting system design will be explained first, covering the tripod-based camera support system that provides stable mounting for the Oak-D Pro camera, the bracket design that ensures proper camera alignment and minimizes vibration during robot movement, and the integration with the existing Stewart platform head that allows synchronized camera and head movements. The fabric integration challenges will be analyzed, including the camera visibility requirements that necessitate fabric modification without compromising Tino's aesthetic appeal, the heat dissipation needs of the Oak-D Pro camera that require ventilation considerations, and the protection requirements that shield sensitive camera components from physical damage during social interactions. The camera shell development will be detailed, covering the custom enclosure design that provides protection while maintaining cooling airflow, the velcro attachment system that secures fabric positioning without interfering with camera operation, and the mesh covering implementation that conceals the camera from casual observation while maintaining full operational capability. The field of view optimization will be examined, including the fabric positioning strategies that prevent interference with camera sensing, the testing procedures that validate optimal camera performance under various fabric configurations, and the reliability evaluation that ensures consistent operation throughout extended social interaction sessions.

\section{Audio System Integration}
This section will present the comprehensive audio system implementation that enables bidirectional communication capabilities for VR integration and enhanced human-robot interaction. The hardware component selection will be detailed first, covering the iTalk-01 omnidirectional microphone specification and mounting considerations within the fabric head structure, the speaker system selection and placement optimization that provides clear audio output without interfering with other robot systems, and the audio processing requirements that enable real-time communication with VR systems. The integration challenges will be analyzed, including the acoustic isolation needed to prevent feedback between microphone and speakers, the cable routing through the robot's structure that maintains mechanical flexibility while ensuring reliable connections, and the power management considerations that integrate audio components with the overall system power budget. The software implementation will be examined, covering the audio\_node.py and audio\_loopback.py ROS2 nodes that handle audio capture and playback, the bidirectional communication protocols that enable seamless VR audio integration, and the audio processing algorithms that ensure high-quality sound transmission and reception. The performance validation will be discussed, including audio quality testing that demonstrates suitable performance for human-robot communication, latency measurements that verify real-time communication capabilities, and integration testing that validates seamless operation with the VR system and overall robot behavior control.

\section{Mechanical Reliability Improvements and Testing}
This section will document the systematic approach to identifying and resolving mechanical reliability issues that affected the legacy Tino system and the validation procedures used to ensure improved performance in Tino V2. The failure analysis methodology will be explained first, covering the systematic documentation of component failures during development and testing, the root cause analysis procedures that identified design weaknesses and operational stress factors, and the prioritization of improvements based on criticality and impact on robot operation. The wheel system improvements will be detailed, including the plastic wheel hub failure analysis that led to hot glue reinforcement as an interim solution, the tire de-beading issues caused by robot weight and the hot glue filling solution that restored proper tire-to-rim interface, and the wheel bumper implementation that prevents fabric entanglement during robot movement. The structural enhancements will be examined, covering the aluminum profile framework that provides improved rigidity and adjustability compared to the original design, the motor bracket modifications that ensure proper alignment and reduce mechanical stress, and the fastener and connection improvements that enhance overall system reliability. The validation testing procedures will be discussed, including the systematic load testing that verifies component performance under operational conditions, the endurance testing that demonstrates sustained operation capabilities, and the performance monitoring that tracks system health during extended operational periods. Finally, the preventive maintenance protocols will be presented, covering the inspection procedures that enable early detection of potential issues, the component replacement schedules that prevent unexpected failures, and the documentation systems that track system performance and maintenance history for continuous improvement of mechanical reliability.


% Chapter 6: Human Detection and Pose Estimation
\chapter{Human Detection and Pose Estimation}
\section{YOLOv11 Pose Detection Implementation with TensorRT Optimization}
This section details the implementation of YOLOv11 pose detection system optimized for real-time performance on the NVIDIA Orin Nano platform using pre-trained models. The YOLOv11 architecture selection will be explained first, covering the advantages of using the latest YOLO iteration for pose estimation tasks, including improved accuracy in detecting multiple humans simultaneously and optimized network architecture that balances detection accuracy with computational efficiency for embedded platforms. The pre-trained model utilization will be detailed, including the selection of the yolo11n-pose.pt model that provides an optimal balance between accuracy and computational requirements, the model format conversion from PyTorch (.pt) to ONNX (.onnx) format for cross-platform compatibility, and the final optimization to TensorRT engine (.engine) format that maximizes inference performance on the Orin Nano's GPU. The TensorRT optimization process will be examined, covering the engine generation procedures that optimize the neural network for the specific hardware platform and the memory allocation strategies that ensure efficient GPU utilization during real-time operation. The implementation architecture will be discussed, including the ROS2 node design that subscribes to camera topics from the Oak-D Pro and publishes human detection results, and the message publishing system that provides skeleton joint information with confidence scores for other system components.

\section{Stereo Depth Integration for 3D Human Positioning}
This section will present the integration of stereo depth information with 2D pose detection to achieve 3D human positioning capabilities using the Oak-D Pro camera system. The depth data utilization will be detailed first, covering how the stereo camera system provides depth information at detected keypoint locations, the depth value extraction process that determines 3D coordinates for each detected joint, and the coordinate system transformation from camera frame to robot coordinate frame. The 3D positioning methodology will be explained, including the process of combining 2D joint detections with corresponding depth values to create 3D skeleton representations and the real-time processing requirements that maintain system responsiveness while providing human positioning information. The practical implementation will be discussed, including how depth measurement works at varying distances and the integration with the robot's localization system to provide human positions relative to the robot's coordinate frame.

\section{Real-time Skeleton Tracking with 17 Key Body Joints}
This section will detail the skeleton tracking implementation that extracts and processes 17 standard COCO keypoints for human posture detection. The keypoint detection framework will be explained first, covering the 17 keypoints detected by YOLOv11 including nose, eyes, ears, shoulders, elbows, wrists, hips, knees, and ankles, and the confidence scoring system that indicates detection reliability for each joint. The data processing pipeline will be detailed, including the extraction of keypoint coordinates and confidence scores from YOLOv11 output, the organization of joint information into structured skeleton representations, and the real-time publishing of skeleton data through ROS2 topics. The message format and data flow will be discussed, including the custom ROS2 message structures that transmit skeleton data with timestamps and the integration with other system components that can utilize human pose information for robot operation and VR data recording.

\section{Performance Optimization and Real-time Operation}
This section will present the optimization strategies implemented to achieve real-time performance of the human detection system on the Orin Nano platform. The computational optimization techniques will be detailed first, covering the TensorRT engine utilization that maximizes GPU inference performance and the memory management strategies that prevent resource exhaustion during continuous operation. The real-time performance characteristics will be explained, including the frame rate capabilities achieved by the optimized system and the processing latency from camera input to pose detection output. The system integration optimization will be examined, covering how the pose detection system operates alongside other robot functions including SLAM, localization, and movement control without creating performance bottlenecks. The practical performance evaluation will be discussed, including testing under various operational scenarios and system stability during extended operation periods.

\section{Integration with System Architecture}
This section will detail the integration of human detection with Tino's overall system architecture and coordination with other robot components. The data flow architecture will be explained first, covering how human pose detection results are published through ROS2 topics and made available to other system components including robot controller nodes and navigation systems. The system coordination will be discussed, including how pose detection operates in parallel with other robot functions such as localization, movement control, and audio processing without creating performance bottlenecks. The ROS2 integration implementation will be examined, covering the message publishing system that provides skeleton joint information with timestamps and the node architecture that ensures reliable data transmission to other system components. Finally, the practical benefits will be addressed, covering how real-time human detection enhances the robot's operational awareness and enables improved human-robot interaction capabilities through better understanding of human presence and positioning.


% Chapter 7: VR Integration and Movement Control
\chapter{VR Integration and Atomic Movement System}
\section{VR System Architecture and Unity Communication}
This section will detail the comprehensive VR integration system that enables remote control and monitoring of Tino through Unity-based VR environments. The VR interface architecture will be explained first, covering the ROS2 \texttt{vr\_interface\_node} that serves as the central communication bridge between the robot's ROS2 system and external Unity applications, and the UDP communication protocol that provides real-time bidirectional data exchange for low-latency VR interaction. The Unity integration capabilities will be detailed, including the message structures for sending robot control commands from VR to ROS2 topics, the data reception system that provides robot pose, human detection, and audio information to Unity for visualization and interaction, and the networking configuration that enables flexible deployment across different network environments. The communication monitoring system will be examined, covering the configurable send rates for pose and skeleton data transmission, the health monitoring that tracks communication status and detects connection failures, and the message ordering system that ensures reliable data delivery and duplicate detection. The VR data recording functionality will be discussed, including the comprehensive recording system that captures all VR-relevant data streams for offline analysis.

\section{Atomic Movement System Design and 4-State Control Architecture}
This section will present the revolutionary atomic movement system designed specifically for natural VR interaction, replacing the previous continuous control scheme with discrete, completion-guaranteed movements. The 4-state control framework will be explained first, covering the unified state architecture applied to both leg and base controllers where state 0 represents idle/resting position, state 1 implements expressive ``little push'' movements for attention-getting behaviors, state 2 provides timing synchronization cycles, and state 3 executes atomic movements that must complete before new commands can be processed. The leg controller implementation will be detailed, including the state 1 optimized 3-phase movement (50\% forward extension, 5\% pause, 45\% return), the state 2 forward extension to maximum reach with position locking mechanisms, and the state 3 return-to-neutral movement with button-press completion detection. The base controller design will be examined, covering the state 1 rapid forward-backward sequence for expressive pointing behaviors, the state 2 timing cycle that provides 1.5-second synchronization delay, and the state 3 atomic movements including forward translation and left/right rotation operations, each with 1.7-second execution duration. The synchronization architecture will be discussed, including the sophisticated locking system that prevents base state 3 execution until leg state 2 completion, and the pending command system that stores VR commands during ongoing operations and automatically executes them upon completion.

\section{Pulse-Based Command System for VR Integration}
This section will detail the pulse-based command architecture that ensures perfect correspondence between VR user actions and physical robot movements. The pulse generation system will be explained first, covering the replacement of continuous signal transmission with discrete 3-cycle command pulses that automatically return to idle state, ensuring each VR interaction triggers exactly one complete robot movement cycle. The gamepad integration modifications will be detailed, including the removal of analog joystick control in favor of discrete button-based state commands, and the implementation of pulse timing that provides consistent command duration regardless of user input duration. The VR command processing will be examined, covering the UDP packet structure that transmits head control data (pitch, pan, tilt), base movement commands (state and angular direction), and audio parameters (volume and orientation), all synchronized with message ordering for reliable delivery. The atomic guarantee system will be discussed, including the movement completion assurance that prevents partial operations, the state machine locks that maintain movement integrity, and the natural interaction flow that ensures VR users always observe complete robot actions rather than interrupted movements. The timing optimization will be addressed, covering the precise 1.5-second state 2 timing cycle, the 1.7-second state 3 movement duration, and the synchronization mechanisms that coordinate multi-component movements for realistic dragging simulation.

\section{Unity-ROS2 Communication Protocol and Message Structures}
This section will present the comprehensive communication protocol designed for robust Unity-VR to ROS2 integration with optimal performance and reliability. The UDP communication architecture will be explained first, covering the multi-port configuration with port 5005 for incoming VR commands, port 5006 for outgoing robot pose data, and port 5007 for human skeleton transmission, enabling parallel data streams without interference. The incoming message format will be detailed, including the 32-byte VR command packets containing 3 floats for head control (pitch, pan, tilt), 2 integers for base commands (state 0--3, angular direction $-1/0/1$), 2 values for audio control (volume and orientation), and 1 integer for message ordering to detect lost or duplicate packets. The outgoing data structures will be examined, covering the 24-byte robot pose packets with position and orientation data fused from UWB and RTAB-Map systems, and the 208-byte skeleton packets containing exactly 17 COCO-format joints with consistent 3D coordinates for missing or occluded body parts. The configurable transmission rates will be discussed, including independent control of pose data frequency (default 10Hz), skeleton data frequency (default 10Hz), and expected incoming command rate (default 25Hz) to optimize performance for different network conditions and VR application requirements. The monitoring and debugging capabilities will be addressed, covering the comprehensive logging system that tracks communication health, the rate validation that ensures expected data flow, and the error detection mechanisms that identify connection problems and provide detailed diagnostic information for system maintenance.

\section{Bidirectional Audio Communication and Spatial Processing}
This section will detail the advanced audio communication system that enables natural voice interaction between VR users and the physical robot environment. The audio data flow architecture will be explained first, covering the microphone input processing that captures robot-side audio and transmits it to VR systems through ROS2 topics, the VR audio reception that provides spatial audio information with volume and orientation parameters for immersive sound positioning, and the bidirectional communication that enables real-time voice interaction between VR users and people in the robot's physical environment. The audio processing implementation will be detailed, including the 16-bit PCM audio sample handling through Int16MultiArray message structures, the real-time audio streaming that maintains low latency for natural conversation flow, and the volume and orientation control system that allows VR applications to adjust audio characteristics based on virtual positioning and interaction context. The spatial audio integration will be examined, covering the orientation parameter system that provides directional audio information in degrees, the volume control mechanisms that enable distance-based audio attenuation simulation, and the Unity integration capabilities that support immersive audio experiences in VR environments. The practical applications will be discussed, including the human-robot interaction enhancement through voice communication, the remote presence capabilities that allow VR users to participate in physical environment conversations, and the research data collection features that record audio interactions for analysis of human-robot communication patterns and social interaction behaviors.

\section{VR Data Recording and Research Integration}
This section will present the comprehensive VR data recording system designed for research applications and offline VR development. The data recording architecture will be explained first, covering the \texttt{vr\_data\_recorder\_node} that subscribes to all VR-relevant topics including robot pose, human detection, skeleton tracking, and audio streams, the SQLite database storage system that efficiently captures timestamped message data for comprehensive interaction analysis, and the recording control services that enable start\slash{}stop functionality for targeted data collection sessions. The Unity integration tools will be detailed, including the data extraction utilities that convert ROS2 message data into Unity-compatible JSON formats, the offline playback capabilities that enable VR development and testing without requiring live robot connection, and the message structure preservation that maintains full fidelity of robot sensor data for accurate VR simulation. The research applications will be examined, covering the human-robot interaction analysis enabled by synchronized recording of human pose detection, robot movements, and audio communication, the VR user behavior studies that analyze interaction patterns and command sequences, and the system performance evaluation that tracks communication rates, latency, and reliability metrics across different operational scenarios. The development workflow benefits will be discussed, including the VR application testing capabilities that use recorded data for consistent development environments, the debugging tools that enable analysis of communication problems and timing issues, and the educational applications that provide realistic robot interaction data for VR training and demonstration purposes. Finally, the extensibility features will be addressed, covering the modular recording system that can be configured for specific research requirements, the data format compatibility that supports integration with external analysis tools, and the scalability considerations that enable recording of extended interaction sessions for longitudinal studies.


% Chapter 8: System Evaluation, Future Work and Conclusions
\chapter{System Evaluation, Future Work and Conclusions}
\section{Localization System Performance Evaluation}
This section will present the evaluation results of Tino's localization systems, comparing pure SLAM performance with the implemented UWB sensor fusion approach. The RTAB-Map SLAM baseline evaluation will be detailed first, covering the systematic testing conducted at four fixed positions with multiple iterations, the significant drift issues discovered with maximum deviations of 1.20 meters, and the need for manual relocalization through spinning movements that proved impractical for VR users without technical expertise. The UWB sensor fusion implementation assessment will be examined, including the positioning system integration that provides improved accuracy with position coordinates showing better consistency across multiple trials, the RTAB-Map orientation preservation that maintained reliable heading information, and the significant improvement in positioning stability with the combined UWB position and SLAM orientation approach. The experimental methodology will be discussed, covering the controlled testing environment with marked floor positions for measurement points, the multiple iteration protocol that provided comparison data, and the systematic data collection that revealed both system capabilities and limitations. The performance comparison analysis will be addressed, including position accuracy measurements showing UWB coordinates with improved consistency compared to pure SLAM, orientation reliability maintained through RTAB-Map integration, and the practical implications for VR integration requiring more reliable positioning without manual intervention.

\section{System Limitations and Identified Challenges}
This section will provide a comprehensive analysis of the identified system limitations and technical challenges encountered during development and testing. The localization system constraints will be detailed first, covering the RTAB-Map dependency on visual features that requires careful environment preparation with sufficient landmarks, the drift accumulation issues that necessitated the UWB sensor fusion implementation, and the map alignment challenges that require manual angle offset adjustments for proper Unity integration. The hardware reliability concerns will be examined, including the mechanical stress points identified during testing such as wheel hub failures requiring temporary repairs, the power system limitations that affect extended operation capabilities, and the component integration challenges that impact overall system robustness. The computational resource limitations will be discussed, covering the NVIDIA Orin Nano platform constraints that require careful optimization of concurrent processes, the memory management requirements for simultaneous SLAM, pose detection, and VR communication operations, and the thermal considerations during extended operation periods. The network communication challenges will be addressed, including the UDP packet loss scenarios that require robust error handling, the bandwidth limitations when transmitting high-frequency pose and skeleton data, and the latency variations that affect VR user experience quality. The VR system integration constraints will be covered, including the Unity development complexity for implementing robust robot communication, the synchronization challenges between virtual and physical environments, and the user experience limitations when technical expertise is required for system recovery or troubleshooting.

\section{Future Improvements and Research Directions}
This section will outline the comprehensive roadmap for future system enhancements and research opportunities identified through the evaluation process. The localization system enhancements will be detailed first, covering the potential integration of additional UWB anchors for improved positioning accuracy and coverage area expansion, the exploration of advanced sensor fusion techniques combining IMU data with existing UWB and visual systems, and the development of automatic map alignment algorithms that eliminate manual angle offset requirements. The mechanical system improvements will be examined, including the replacement of failure-prone components with more robust alternatives, the implementation of predictive maintenance capabilities through sensor monitoring, and the exploration of advanced actuator systems that provide smoother movement profiles and reduced mechanical stress. The computational optimization opportunities will be discussed, covering the potential migration to more powerful embedded platforms as they become available, the implementation of distributed processing architectures that leverage multiple computational units, and the development of adaptive resource management systems that dynamically allocate processing power based on operational requirements. The VR integration advancement possibilities will be addressed, including the development of more sophisticated interaction paradigms that leverage advanced pose detection capabilities, the implementation of haptic feedback systems that enhance user immersion and control precision, and the exploration of multi-user VR environments that support collaborative robot interaction scenarios. The research application extensions will be covered, including the development of comprehensive datasets for human-robot interaction research, the implementation of machine learning systems that adapt robot behavior based on user interaction patterns, and the exploration of social robotics applications that leverage the established VR integration framework. Finally, the long-term vision will be presented, covering the potential evolution toward fully autonomous social interaction capabilities, the integration with broader smart environment systems, and the development of replicable platforms that enable wider adoption of VR-integrated social robotics research.

\section{Conclusions and Lessons Learned}
This section will synthesize the key findings from the Tino V2 development process, highlighting significant achievements and valuable insights for future robotics research. The technical achievement summary will be presented first, covering the successful migration from legacy Raspberry Pi architecture to modern ROS2-based systems on NVIDIA Orin Nano, the implementation of sensor fusion combining UWB positioning with RTAB-Map orientation that addresses localization drift issues, and the development of atomic movement systems that enable natural VR interaction through discrete, completion-guaranteed operations. The methodological insights will be detailed, including the importance of systematic testing protocols for identifying system limitations like SLAM drift issues, the value of modular architecture design that enables independent development of localization, movement, detection, and communication subsystems, and the critical role of practical testing in revealing real-world challenges such as mechanical reliability concerns. The interdisciplinary integration lessons will be examined, covering the challenges and benefits of combining computer vision, robotics, networking, and VR technologies in a cohesive system, the importance of maintaining real-time performance requirements across all subsystems, and the value of comprehensive documentation and monitoring systems for troubleshooting complex multi-component interactions. The research contribution significance will be discussed, including the advancement of VR-integrated social robotics through practical implementation of bidirectional communication systems, the development of atomic movement architectures that ensure predictable robot behavior for remote users, and the demonstration of sensor fusion approaches that combine complementary positioning technologies. The broader implications will be addressed, covering the potential impact on social robotics research through demonstrated VR integration capabilities, the contribution to human-robot interaction studies through comprehensive data recording and analysis tools, and the advancement of embedded robotics development through optimization of complex algorithms on resource-constrained platforms. Finally, the future outlook will be presented, covering the foundation established for advanced social interaction research, the potential for broader adoption of VR-integrated robotics platforms, and the continued evolution toward more sophisticated and accessible human-robot interaction systems that bridge physical and virtual environments effectively.


% \chapter{Temporal R\&D}
\label{ch:chapter_one}

\section*{Localization Technologies}

\subsection*{Onboard Sensing}
\begin{table}[H]
    \centering
    \begin{tabular}{|>{\raggedright\arraybackslash}p{3cm}|>{\raggedright\arraybackslash}p{3cm}|>{\raggedright\arraybackslash}p{3cm}|>{\raggedright\arraybackslash}p{5cm}|}
        \hline
        \textbf{Technology} & \textbf{Pros} & \textbf{Cons} & \textbf{Key Papers \& Resources} \\ \hline

        Visual Odometry (VO) & Could use existing camera; no hardware mods& Narrow FOV; tilt disrupts SLAM & \cite{campos2021orbslam3} – Robust monocular/Stereo SLAM. \\ 
         & & & \cite{forster2016svo} for Monocular and Multicamera Systems \\ \hline
  
        IMU + Wheel Encoders & Low cost; integrates motion data & Drift over time; Stewart tilt issues & \cite{huang2018sensor} – Kalman filtering. \\ 
         & & & \cite{ekf2021practice} for Mobile Robot Localization \\ \hline
         
        UWB-IR & Small footprint; could work with fabric & Requires external anchors & \cite{uwb2007localization} for Indoor Robot Navigation. \\ \hline
    \end{tabular}
\end{table}



 
\subsection*{External Sensing}
\FloatBarrier

\begin{table}[H]
    \centering
    \begin{tabular}{|>{\raggedright\arraybackslash}p{3cm}|>{\raggedright\arraybackslash}p{3cm}|>{\raggedright\arraybackslash}p{3cm}|>{\raggedright\arraybackslash}p{5cm}|}
        \hline
        \textbf{Technology} & \textbf{Pros} & \textbf{Cons} & \textbf{Key Papers \& Resources} \\ \hline
        UWB Anchors & High accuracy; no line-of-sight & Setup/calibration required & \cite{chen2018uwb} in non-cooperative industrial environments  \\ \hline
        
        AprilTags & Low cost; precise & Line-of-sight; limited area & \cite{olson2011apriltag}. \\ \hline
        
        MoCap Systems & Sub-mm accuracy & Expensive; fixed environment & \cite{optitrack2024robotics} – Industrial use cases. \\ \hline
    \end{tabular}
\end{table}


\section*{Orientation Technologies}
\begin{itemize}
    \item \textbf{Sensor Fusion}: \cite{huang2018sensor} filters for combining UWB, IMU, and encoders. (Waiting for access request)

    \item \textbf{UWBOri}: \cite{uwbori2024} with ultra-wideband signals 

    \item \textbf{NLOS Mitigation}: \cite{luo2020uwb} and Tracking With Arbitrary Target Orientation, Optimal Anchor Location, and Adaptive NLOS Mitigation

    \item \textbf{RPO}: \cite{rpo2019uwb} using UWB
\end{itemize}

\section*{Human Detection Technologies}
\subsection*{Onboard Sensing}

\begin{table}[H]
    \centering
    \begin{tabular}{|>{\raggedright\arraybackslash}p{3cm}|>{\raggedright\arraybackslash}p{3cm}|>{\raggedright\arraybackslash}p{3cm}|>{\raggedright\arraybackslash}p{5cm}|}
        \hline
        \textbf{Technology} & \textbf{Pros} & \textbf{Cons} & \textbf{Key Papers \& Resources} \\ \hline
        Thermal Cameras & Works in darkness; fabric-friendly? & No depth; limited range & \cite{thermal2019detection} – CNN-based approaches. \\ \hline
        Ultrasonic Array & Low cost; proximity detection & No human distinction & \cite{ultrasonic2020tracking} (In Korean). \\ \hline
        Upgraded Camera & Wider FOV; ML-compatible & Fabric obstruction; compute-heavy & \cite{wang2022yolov7} – Real-time object detection. \\ \hline
    \end{tabular}
\end{table}

\subsection*{External Sensing}

\begin{table}[H]
    \centering
    \begin{tabular}{|>{\raggedright\arraybackslash}p{3cm}|>{\raggedright\arraybackslash}p{3cm}|>{\raggedright\arraybackslash}p{3cm}|>{\raggedright\arraybackslash}p{5cm}|}
        \hline
        \textbf{Technology} & \textbf{Pros} & \textbf{Cons} & \textbf{Key Papers \& Resources} \\ \hline
        RGB-D Cameras & Depth data; multi-human tracking & Fixed installation & \cite{cao2019openpose}. \\ 
         & & & \cite{azure2021kinect} – Performance Analysis of Body Tracking. \\ \hline
        LiDAR & High-resolution 3D mapping & Expensive; compute-heavy & \cite{yan2019lidar}. \\ \hline
        WiFi/Radar & Privacy-friendly; fabric-penetrating & Lower resolution & \cite{rfsensing2022}: A New Way to Observe Surroundings\\ \hline
    \end{tabular}
\end{table}


\section*{Technologies for Tino Robot Implementation}

\subsection*{Localization Technologies}

\begin{description}
\item[\textbf{Visual Odometry (VO)}]
\hfill
\begin{itemize}[leftmargin=*,nosep]
    \item \textbf{Variants:}
    \begin{itemize}
        \item \textit{ORB-SLAM3} (supports RGB-D): \cite{orbslam3github}
        \begin{itemize}
            \item[+] Synergy with human detection via depth data
            \item[+] Robust feature matching for dynamic environments
            \item[--] Higher computational cost (requires GPU optimization)
        \end{itemize}
        \item \textit{SVO} (Semi-direct Visual Odometry): \cite{svogithub}
        \begin{itemize}
            \item[+] Works with fisheye/catadioptric cameras (wide FOV)
            \item[+] Lower computational footprint
            \item[--] Less accurate in textureless environments
        \end{itemize}
    \end{itemize}
    \item \textbf{Shared Advantage:} Dual-purpose for localization \& human detection
\end{itemize}

\item[\textbf{UWB-IR Localization}]
\hfill
\begin{itemize}[leftmargin=*,nosep]
    \item[+] Centimeter-level accuracy (theoretical)
    \item[+] Low power consumption
    \item[--] Requires external infrastructure (anchors)
    \item[--] Fabric penetration uncertainty (needs RF testing)
    \item[--] No native orientation data $\Rightarrow$ Requires:
    \begin{itemize}
        \item IMU sensor fusion (Kalman filtering)
        \item RPO/UWBOri techniques (experimental)
        \item NLOS mitigation strategies
    \end{itemize}
\end{itemize}

\item[\textbf{Wheel Encoders + IMU}]
\hfill
\begin{itemize}[leftmargin=*,nosep]
    \item[+] Low-cost solution
    \item[--] Unsuitable for impulse-based movement (slippage errors)
    \item[--] IMU drift accumulates over time
    \item[--] Poor performance on uneven surfaces
\end{itemize}
\end{description}

\subsection*{Human Detection Technologies}

\begin{description}
\item[\textbf{RGB-D Camera (e.g., Intel RealSense)}]
\hfill
\begin{itemize}[leftmargin=*,nosep]
    \item[+] Simultaneous color + depth data
    \item[+] Enables skeleton tracking (OpenPose, MediaPipe)
    \item[--] Requires careful physical integration (size/visibility)
    \item[--] Limited range (typically <5m)
\end{itemize}

\item[\textbf{Thermal Imaging}]
\hfill
\begin{itemize}[leftmargin=*,nosep]
    \item[+] Potential fabric penetration capability
    \item[+] Works in low-light conditions
    \item[--] No depth sensing $\Rightarrow$ Requires fusion with VO
    \item[--] Limited contextual information (heat-only data)
\end{itemize}

\item[\textbf{ML-Enhanced 2D Camera}]
\hfill
\begin{itemize}[leftmargin=*,nosep]
    \item[+] Lower profile than RGB-D
    \item[+] Modern architectures (YOLOv8, EfficientNet) enable real-time detection
    \item[--] Requires depth estimation via:
    \begin{itemize}
        \item Monocular depth networks (MiDaS, LeReS)
        \item Sensor fusion with other localization data
    \end{itemize}
\end{itemize}

\item[\textbf{Lidar}]
\hfill
\begin{itemize}[leftmargin=*,nosep]
    \item[--] Impractical due to Tino's soft structure (vibration issues)
    \item[--] High cost-to-benefit ratio
    \item[--] Overkill for indoor social robot ranges
\end{itemize}
\end{description}

\subsection*{Recommended Hybrid Approach}
\begin{itemize}
\item \textbf{Localization:} ORB-SLAM3 with RGB-D camera (despite computational cost) + optional UWB for absolute positioning
\item \textbf{Human Detection:} Thermal camera + RGB-D fusion (if concealable) or ML 2D camera with monocular depth estimation
\item \textbf{Backup:} SVO with fisheye lens as fallback if RGB-D integration fails
\end{itemize}




Week 18 Mar
R\&D on different techs

Week 25 Mar
Task: Work on Orin nano testing cameras ZED 2, and Orb slam 3 with webcam and Realsense T265
Result: The Zed camera that was available was not functioning properly, orbslam had a lot of bugs in terms of compilation given is an old library that is not being maintained.

Week 1 Apr
Task: Work on Orin Nano and Relsense T265 to try and make SLAM atlas creation and load
Result: The T265 was deprecated so I had to install an old version of librealsense (2.53) in order to make the camera be detected, even after camera detection I was able to run the camera with orbslam but the accuracy was very low, my initial tought is that it was because of poor calibration. Orbslam had some issues with the camera, in stereo inertial was the best mode that it worked but it needed some acceleration in order to start outputing some video, also it took a long time to actually grap into something (features) so to actually start creating a map, in only stereo it crashed, same as in Mono

Week 08 Apr
Task: keep working on Realsense T265 and most important save and load atlas
Result: even tho it had a lot of issues the first thing that was tested is calibrating the camera to see if the detection improved, it didnt, then i tried saving the atlas but given the old library it always ended in a crash. After looking on the web I found a git fork that "fixed" this atlas save and load, testing the library i found that it managed to save but it always crashed when trying to open the atlas back, either that or it starts creating a new map from scrathc. 
Given all of the issues that the orbslam3 had i tried using SVO, it had again a lot of issues given its an old not maintained library, mainly in the compilation part as i am working in arm64 so i had to fix a lot of flags in order to make it compile in arm64. even after all of the work trying to make it build in a container i had a same result that with orbslam, loaded the map, mapped something (not that accurrate) and they did not have any atlas/map management so I scraped that work.
Given that with 2 systems i had simmilar issues i tought it could be related to the Realsense T265, so I requested if a D435I was available (given that the video examples used in orbslam3 are with that camera), in the end that camera was not available but I was provided with a oak-d pro, after testing the basic functionallity with the depthAi library I tried checking for slam approaches, they had a community fork of orbslam3 using that camera and a guide to try and build it with lxc, but after trying a lot I was not able to pass the camera to the container in a way that it was detected as a bootable device. One of the other options Luxonics mentrioned was using RtabMap, so thats what I was set to try 

Week 15 Apr
Task: Try to set the Oak-D pro to work with Rtabmap
Result: The first thing I had to do was try to build the library, it had a lot of dependencies and with that a lot of issues to be fixed, the first time I tried to build the standalone version, this didnt work at first because the depthAI library was not detected.
Then I tried to build the Ros version of rtab but this one had not a proper implementation between depthAI and the ros wrapper

Then I tied again with the standalone version and this time I was able to make it link with the depthAI, and it worked amaizing, I did some tests with the camera over Tino moving around, this showed that Rtab was working really well creating a map and most importantly it was able to save a map and then relocalize itself in that map.
Now the next step was how to get the data out of the standalone version, how to get the position and orientation.

After some trial an error managed to install and run it with ros2 using the depthai\_ros and the rtabmap\_ros, it publishes the localization\_pose topic that has all of the importnat information

Managed to load the correct map, I refactored the old Tino source code to work with Ros2, created the respective topics and the needed structure. Created respective launch files for mapping and localization modes
Added human detection, this system works by subscribing to the same camera topic and run it with yolo11 in tensorRt format. This provides all of the information needed for the human detection getting all of the skeleton pose joints, getting the depth (using the stereo camera info) and position in relation to the robot.

Also by creating this ros version I created a node for handling the VR connection in the future. 

Apr 19
Major milestone reached: Complete system architecture migration from legacy Raspberry Pi based system to ROS2 based implementation. Restructured the entire workspace by creating tino\_ws for ROS2 development and moved all legacy code to legacy\_tino folder for preservation.

Implemented the core ROS2 node architecture:
\begin{itemize}
\item gamepad\_node.py: Handles Xbox controller input with proper D-input to X-input conversion for Jetson compatibility
\item hardware\_interface\_node.py: Manages serial communication with all 3 Arduino systems (head, base, leg) using proper device symlinks
\item robot\_controller\_node.py: Central coordination node that manages all robot behaviors and movement commands
\item vr\_interface\_node.py: Handles VR system integration and data exchange for future Unity integration
\end{itemize}

Created launch files for both mapping (rtab\_mapping.launch.py) and localization (rtab\_localization.launch.py) modes, allowing seamless switching between SLAM creation and navigation modes.

This migration allows for much better modularity, debugging capabilities, and integration with the VR system compared to the monolithic Python scripts from the legacy system.

Apr 22
Next task was adding audion in/out to the system. I was provided with a omnidirectional mic iTalk-01, and a pair of speakers. Impelemented a system that gets the data and publishes it to the vr, and also receives from the vr and publishes to the speakers.

Apr 23
Major advancement in human detection capabilities: Implemented pose detection functionality using YOLOv11 with TensorRT optimization. The system now provides real-time human skeleton tracking with 17 key body joints detection, including depth information using the stereo camera data. This allows Tino to not only detect humans but also track their pose and calculate their 3D position relative to the robot.

Added audio handling nodes (audio\_node.py and audio\_loopback.py) and fully integrated audio functionality into both VR and robot controller systems. The audio system now supports bidirectional communication: capturing audio from the omnidirectional microphone and publishing it to VR, while also receiving audio from VR and playing it through the speakers.

Enhanced gamepad handling with improved command processing and error reporting, making the control system more robust and responsive.

At this point most of the internals where ready
We bought a display port dummy in order to have good performance when connected via vnc because the Orin Nano does not run headless by default

One of the head supports broke so we had to print a new one with more internal support

Next steps is hardware related. we need to:
Build the new kinematic base that can support tino weight
Fix the head supports
Add the power supply needed to support the Orin Nano


Apr 29
Started by dissasembling the robot completely into the main 4 parts
Fabric head
Servo Head
Body
Kinematic base

First I modified the Servo head by adding a trypod that can hold the camera, this was done with simple brackets to make the support fixed and steady, specially because the old camera mount (that was for a pi camera) was very very flexible and moved a lot
Then I tested the power supply, we got a powerful and stable 12v to 19v DC DC step up converter Oumefar, using this proved and testing the Orin Nano at max power, so with all of tino system actives (SLAM, audio, ROS) it reached a max of 2A of consumption, this from a 12v battery They are 5200 mAh 80c 11.1v 57.72Wh gave a approximate time of 1.37 hours during max consumption, but this really is not accurate as the jetson usually works between 1.3 to 1.4 A
https://chatgpt.com/c/68125c25-216c-8000-a956-52b2702d04b8

Given that this will fix the power supply issue I modified the cable harness to remove the old USBA and USBC that powered the Raspberry from a powerbank, and replaced it with the 12V input and the 19V DC jack the Orin needs
Also we added a 12v to 5V converter connected to the same 12v battery to power the Onboard router and the Oak-D camera
the camera can be powered by the orin but we wanted to leave the option to power it directly if we wanted in the future to add the machine learning algorithms inside the camera.
Also doing this change helped us remove the powerbank that was dedicated to the router, helping the total process of turning tino on and reducing from 4 batteries to 3 batteries


(couldn't do more because the week had thursday and Friday as holiday)

May 6
This week started on upgrading the kinamtic base. given that tino had an omniwheel base and tino is almost 20KG the wheels that it have where breaking apart and getting stuck, the rollers of the wheels where getting squared out. this also because given the movement, the back wheel of the triksta base was most of the time being dragged, as tino only moves forward and turn side to side

given this issues we decided to remove the omnidirectional triksta base as this movement is not needed, we decided a simple but reliable differential drive system, using 2 wheels at the front and a caster wheel in the back. To start this process we decided to just modify the base instead of changing it, given that it already had most of the things we needed.
We removed the 3 motors, replaced them with 2 more powerful motors, given this new motors whe changed the old 2 motor drivers with a new and more powerful mdd10a.

We built the T structure using Aluminium profiles, item, this allowed us to have a dynamic and regulable system where we can extend out the wheels to try and get a proper balance.
One of our main issues where the wheels we started by using plastic wheels that had a rubber neumatic, this worked first but then when tino was built it created an issue

May 8
Completed major refactor of serial communication and motor control systems for the new differential drive base implementation. Created new\_base\_tino.ino with enhanced PID controller specifically designed for the differential drive system, replacing the old omnidirectional control logic.

Updated hardware\_interface\_node.py with improved serial port configurations, added comprehensive debugging logs, and enhanced command handling to work with the new base architecture. The new system uses proper device symlinks (/dev/ttyBASE, /dev/ttyHEAD, /dev/ttyLEG) to ensure consistent Arduino connections.

Created upload scripts (upload\_new\_base.sh, setup\_arduino\_symlinks.sh) for easier development and deployment workflow.

Once the new base was rebuilt I had to modify the code for it to work, the original base used the VirHas library (custom internal library of the airlab to manage and control the triksta bases) so given this used a differential drive I had to implement my own PID movement controller (Proportional–integral–derivative controller) but keeping the commands the same in order to keep the original tino movement

Once the base was ready I started rebuilding tino, removing things that where not needed and adding the new things and new cable harness. I also added the speakers in the servo head because it had enough space and the microphone was passed through the fabric on the head

Then, once built, I had to test the connection from the arduinos to the Jetson, this proved to have some issues, the head Arduiono was a az-delivery arduino mega clone, but the jetson does not had the needed drivers (CH340) the only solution was to rebuilt the kernel of the jetpack system so to include it, given that this would be time consuming I decided to change that arduino mega with a Arduino elego uno r3. this one properly linked and connected to the jetson, the change did not create any issue as the head only used 3 PWM pins for the servos.
I also setup a symlink using the serial of the devices so that when connected they always be in the same route /dev/ttyHEAD /dev/ttyBASE /dev/ttyLEG

once all systems where working again and some fine tunning had to be done to the gamepad (we changed from D input to X input because the jetson did not had the drivers to manage the D input) tino was working once again. The next step was going to test the wheels, the wheels we had put, given the weight of the robot the tire Partially de-beaded. consulting this issue we had 3 approaches to take next week:

\begin{enumerate}
\item Fill the current wheels with hotglue, easy but could cause issues with the traction of the tire
\item use some hard plastic wheels but is demanding in labor because this wheels do not have the 6mm axis needed to connect to the motor axis, so we will need to modify them a lot in order to connect properly
\item buy a new pair of wheels that can support this weight better
\end{enumerate}

We also encountered some issues with the fabric enrolling over the wheels so we may need to add a type of ``bumper'' in order to avoid this

Also we need to find a way to make the camera avoid the fabric, or better said, the fabric to avoid moving over the camera FOV
I tried sticking the fabric to a foam external shell I put over the camera but this velcro was not sticking to the fabric given the camera is behind the leg, and this side of the robot moves the fabric a lot. Also using this foam to make the shell was not ideal because it was absorbing the camera heat and not letting the camera cooldown.

this are the issues to solve by next week


May 13
This week started by trying to solve the wheel issue, we tried to fill the wheels with hot glue, this worked perfectly, the wheels did not de-bead and the traction was good.
Also create a ``bumper'' to avoid the fabric to get stuck in the wheels, this works on most of the scenarios but there are still some cases that it may get stuck, so we will need to keep testing it.

I didnt have time to create the shell for the camera

May 15
Implemented comprehensive VR data recording functionality for Unity integration. Created vr\_data\_recorder\_node.py and vr\_data\_extractor.py to capture and process all robot sensor data, human pose detection, audio streams, and robot state information for VR system development and testing.

The recording system includes service controls for starting/stopping data capture, metadata management, and proper data synchronization across all robot systems. This allows for detailed analysis of robot behavior and human-robot interactions for VR system optimization.

Enhanced VR interface message structure documentation and improved serial port configurations for better reliability during VR data exchange.

May 20
before continuing with the camera shell we had a new issue. The current arms for the head where breaking, this was because the head is too heavy and has very agressive moves, also given the 3-Dof Stewart platform it has a lot of flex in the arms, so we had to change the arms to a more robust design, Our first hypothesis was that the current desing the servo axis was not aligned with the head axis, this could cause the servos to get damaged as the force was being applied to the servo axis and not to the head axis, so we designed a new arm that has the servo axis aligned with the head axis, this way the force is applied directly to the head and not to the servo axis.
This new model was designed in Inventor and printed in PLA. The idea of this pice is that it can hold the arm that goes to the top from the middle and have it properly tight, also aligned the servo axis with this arm axis and then the head axis.
The pieces where printed and fixed to the head, this worked but we still saw some flex in the arms, that we think is acceptable so we will keep testing it.

May 27
This week I started by trying to create the camera shell, this was done by creating a shell that can be attached to the trypod system that was used to hold the camera on tino, this shell needed to have some important features
It needed to of course hold the camera, but it had to had a empty space in the back to allow the camera to cool down, also it needed to have a way to attach it to the trypod system, and finally it needed to have 2 flaps at the top and at the bottom so that we can use them as a point to glue some velcro to hold the fabric in place and avoid it to get in the camera FOV

Jun 3
This week we finalized the camera shell, we attached the velcro to the flaps and the other side of velcro to the fabric by sewing it. Also we glued a mesh so that we can hide the camera and avoid it to be seen.
We tested the camera and it worked perfectly, the fabric did not get in the camera FOV and the camera was able to cool down properly, also the camera was able to see through the fabric so it was able to detect humans.

Also this week the head arms we printed broke, we thought it was because of the 3d printed layer orientation and force applied, so we redesigned the arms to have a more robust design, this time we printed it in the other orientation so that the layers orientation is perpendicular to the force applied, this way we hope that the arms will not break again.

June 24--July 1
This week we got the new brackets for holding the new more powerful motors, this change was done because the old motors suffered a lot of heat due to tino weight. This new ones can support more weight and have a better heat dissipation
the only issue they had is that the old motor bracket was not compatible so we had to wait for the new ones to arrive. this new bracket proved difficult to implement as the holes they had where not aligned with the item profiles, also the motors axis was more up so tino was dragging on the floor
To fix both of this issues I created a spacer with some metal square profiles, this way the motors where aligned with the item profiles and also the axis was at the right height so that tino could move freely.

Also with this new motors I had to redo some cable connections to the encoders, power and driver simplifying the desing and connections

We tested the new motors and they worked perfectly, there was no dragging and the wheels did not de-bead, also the traction was good and the speed was good enough for tino to move around. We only had to tweak a bit the PID values to make it more stable and not overshoot the target position

Enhanced Raspberry communication system with special command processing and periodic status updates in the main loop for improved reliability and debugging capabilities. This allows for better monitoring of Arduino systems and more robust error handling.

Implemented audio messages system with chime notifications, providing auditory feedback for system events and user interactions. This enhances the user experience and provides clear audio cues for various robot states and actions.

July 1
After a period of use the arms for the head broke again, There is still a lot of flex in the arms, so we decided to redesign the arms again, this time we decided to change the approach. Doing some research on Stewart platforms we found that the arms should have rod end (heim joints) on both ends, this way the arms can move freely and not have any flex. The current design was using a bearing on the servo side and a rod end on the head side, this was not ideal as the bearing was not allowing the arm to move freely, so we redesigned the arms to have rod ends on both sides, this way the arms can move freely and not have any flex. The new design was printed in PLA and combined with some metal heim joints, this way the arms can move freely and not have any flex. 
The only issue this created was that when stationary the head had some wobble, this simple because the head is held by the 3 arms that can move thanks to the rod end. We dont think this is a major issue as it may add a bit to the ``expresiiveness'' of tino head movement.

July 8
Based on the VR system I had to modify the current system of movement to better integrate with the VR environment.
The old system using the gamepad was broadcasting a messgage of movement to the 3 arduinos, this was so that the head, leg, and base could move synchronously
But now in the VR system we need to ``independicese'' this 3 system.

The head is controlled bu the VR movement, this already works given the head comand topics, we only had to remove the defined ``routines'' that it had when moving forward and to the sides.
The leg new needs to be controled with 4 different states.
Previously the leg was just resting and when a movmeent command was sent it did a whole continus rotation based on the sinusoidal system that was created originally.
Now because this new system the idea is that the user in the VR need to ``drag'' the same way tino ``drags'' the leg in the real world, so for this we created 4 states:

\begin{enumerate}
\setcounter{enumi}{-1}
\item Resting
\item little push (this was done so that tino can express a ``pointin'' to something or ``looking'' at something to try and get the attention of the user)
\item leg forward (moves the leg to the max reach before doing the dragging movement)
\item leg backward (moves the leg back doing the ``dragging'' movement up to the resting position)
\end{enumerate}

We also had to modify the base movement, this was done by removing the old gamepad movement and implementing this new 4 state system:

\begin{enumerate}
\setcounter{enumi}{-1}
\item Resting
\item little push (the base moves forward and backward a bit very fast to express a ``pointing'')
\item Does nothing, because at this point the leg is just being moved forward to prepare for the dragging
\item move forward (this is the dragging movement, the base moves forward a determided amount of time to simulate the dragging movement)
\end{enumerate}

In order to make all of this systems work properly we had to made them atomic, and also this new states will be sent as pulses instead of continuous movement
This way the user can control tino in a more natural way, and also the user can express more emotions and actions with tino

While doing the leg changes we noticed something intresting, the PWM that the driver gets for negative values for some reason is not the usual range
from  0 to  255 the motor that moves the leg goes forward from very slow to very fast (as intended) but as one would expect the negative values to go backward, it does not, it goes slow at -235 and fast at -1 
This is not an issue as we can create the respective system to make it work with this values, but it is something to keep in mind for the future

July 10--13
Enhanced VR interface with improved odometry checking and loss detection, simplifying the condition logic for more reliable operation. Updated motor speed and angular parameters for improved performance and smoother robot movement.

Added comprehensive logging improvements across pose detection and VR interface nodes, reducing log verbosity while maintaining essential debugging information. Modified skeleton publish rate in robot controller for optimal performance balance between real-time tracking and system resources.

Improved ROS-TCP-Endpoint submodule with enhanced error handling for more robust Unity-ROS2 communication bridge.

July 15
This week we prepared the experiment but we also had to finalize the atomic movement system and implement the complete 4-state management for both leg and base controllers.

The major achievement was completing the synchronization between the leg and base atomic movements. We implemented a sophisticated locking system where the base controller can only execute case 3 (forward movement) after the leg controller has completed case 2 (leg extension). This ensures that Tino's movements are perfectly coordinated - the leg extends first to prepare for dragging, then the base moves forward to simulate the actual dragging motion.

We also implemented a pending command system that was crucial for the VR integration. When the user in VR triggers a movement command while case 2 is still running (leg extending), the system stores the command and automatically executes it once case 2 completes. This allows for natural user interaction without losing commands or creating timing conflicts.

The rotation system was expanded to work atomically as well. We implemented cases (3,1) and (3,-1) for right and left rotation respectively, maintaining the same 1.7-second duration as forward movement for consistency. This means the VR user can trigger rotation movements that are perfectly timed and atomic just like the forward movements.

One of the most important improvements was the pulse system we implemented in the gamepad node. Instead of continuous signals, each button press now generates a 3-cycle command pulse that automatically returns to idle. This pulse approach is essential for the VR system because it ensures each user action in VR corresponds to exactly one complete movement in the physical robot.

We also enhanced the debug system significantly, adding timing logs and state tracking that shows exactly when each phase starts and completes. This was crucial for fine-tuning the 1.5-second case 2 timing cycle and the 1.7-second case 3 movements to ensure perfect synchronization.

The atomic movement system now guarantees that once any movement starts, it must complete fully before another can begin. This prevents partial movements and ensures the VR user always sees Tino complete the action they initiated, creating a much more natural and predictable interaction experience.

As a cleanup task, we also removed the old left joystick analog control since everything now runs through the discrete 4-state button system, making the control scheme consistent with the VR approach.


Added a system to offset the map positionbyt a desired amount of degree, this becauseits difficult to control the alignment of the SLAM map when creating it and some times it can be created with a rotation, this in principle is no problem, but given that we need to map this position directly into unity is best if we have the map aligned with the cartesian plane. This value can be changed but the process needss to be trial and error until we find a desired angle, this may change from map to map

% --------------------------------------------------------------------------
% NUMBERED CHAPTERS % Regular chapters following
% --------------------------------------------------------------------------

%-------------------------------------------------------------------------
%	BIBLIOGRAPHY
%-------------------------------------------------------------------------

\addtocontents{toc}{\vspace{2em}} % Add a gap in the Contents, for aesthetics
\bibliography{Thesis_bibliography} % The references information are stored in the file named "Thesis_bibliography.bib"

%-------------------------------------------------------------------------
%	APPENDICES
%-------------------------------------------------------------------------

\cleardoublepage
\addtocontents{toc}{\vspace{2em}} % Add a gap in the Contents, for aesthetics
% \appendix
% \chapter{Appendix A}
% If you need to include an appendix to support the research in your thesis, you can place it at the end of the manuscript.
% An appendix contains supplementary material (figures, tables, data, codes, mathematical proofs, surveys, \dots)
% which supplement the main results contained in the previous chapters.

% \chapter{Appendix B}
% It may be necessary to include another appendix to better organize the presentation of supplementary material.
\printglossary[type=\acronymtype]

% LIST OF FIGURES
\listoffigures

% LIST OF TABLES
\listoftables

% % LIST OF SYMBOLS
% % Write out the List of Symbols in this page
% \chapter*{List of Symbols} % You have to include a chapter for your list of symbols (
% \begin{table}[H]
%     \centering
%     \begin{tabular}{lll}
%         \textbf{Variable} & \textbf{Description} & \textbf{SI unit} \\\hline\\[-9px]
%         $\bm{u}$ & solid displacement & m \\[2px]
%         $\bm{u}_f$ & fluid displacement & m \\[2px]
%     \end{tabular}
% \end{table}

% ACKNOWLEDGEMENTS
\chapter*{Acknowledgements}
Here you may want to acknowledge someone.

\cleardoublepage

\end{document}
